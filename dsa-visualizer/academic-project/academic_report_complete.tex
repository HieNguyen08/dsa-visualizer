\documentclass[a4paper]{report}
\usepackage[fontsize=13pt]{scrextend}
\usepackage[style=ieee,sorting=none]{biblatex} %Imports biblatex package
\AtEveryBibitem{%
  \clearfield{issn} % Remove issn
  \clearfield{doi} % Remove doi

  \ifentrytype{online}{}{% Remove url except for @online
    \clearfield{url}
  }
}
\addbibresource{refs.bib}
\usepackage{longtable}
\usepackage{amsmath}
\usepackage{indentfirst}
\usepackage{ragged2e}
\usepackage{blindtext}
\usepackage{titlesec}
\usepackage{a4wide,amssymb,epsfig,latexsym,multicol,array,hhline,fancyhdr}
\usepackage{vntex}
\usepackage[english]{babel}
\usepackage{mathptmx}[ptm]
\usepackage[normalem]{ulem}
\usepackage{amsmath}
\usepackage{lastpage}
\usepackage[lined,boxed,commentsnumbered]{algorithm2e}
\usepackage{enumerate}
\usepackage{color}
\usepackage{graphicx}							% Standard graphics package
\usepackage{array}
\usepackage{tabularx}
\usepackage{caption}
\usepackage{multirow}
\usepackage{multicol}
\usepackage{rotating}
\usepackage{graphics}
\usepackage{geometry}
\usepackage{setspace}
\usepackage{epsfig}
\usepackage{tikz}
\usepackage{subcaption}
\usepackage{longfbox}
\usepackage{float}
\usepackage{pdfpages}
\usepackage{listings}
\usepackage{xcolor}

\usetikzlibrary{arrows,snakes,backgrounds,calc}
\usepackage{hyperref}
\hypersetup{urlcolor=blue,linkcolor=black,citecolor=black,colorlinks=true} 
\usepackage{comment}
\usepackage{tocloft}
\usepackage{appendix}
\usepackage{algorithm}
\usepackage{algorithmic}
\renewcommand{\algorithmicrequire}{\textbf{Input:}}
\renewcommand{\algorithmicensure}{\textbf{Output:}}

\RestyleAlgo{ruled}

\SetKwComment{Comment}{/* }{ */}

\DeclareMathAlphabet{\mathcal}{OMS}{cmsy}{m}{n}

% Code listing setup
\definecolor{codegreen}{rgb}{0,0.6,0}
\definecolor{codegray}{rgb}{0.5,0.5,0.5}
\definecolor{codepurple}{rgb}{0.58,0,0.82}
\definecolor{backcolour}{rgb}{0.95,0.95,0.92}

\lstdefinestyle{mystyle}{
    backgroundcolor=\color{backcolour},   
    commentstyle=\color{codegreen},
    keywordstyle=\color{magenta},
    numberstyle=\tiny\color{codegray},
    stringstyle=\color{codepurple},
    basicstyle=\ttfamily\footnotesize,
    breakatwhitespace=false,         
    breaklines=true,                 
    captionpos=b,                    
    keepspaces=true,                 
    numbers=left,                    
    numbersep=5pt,                  
    showspaces=false,                
    showstringspaces=false,
    showtabs=false,                  
    tabsize=2
}

\lstset{style=mystyle}

% \renewcommand{\cftfigindent}{pt}
\renewcommand{\cftfigpresnum}{\figurename~}
\renewcommand{\cftfignumwidth}{5em}

% \renewcommand{\cfttabindent}{0.5pt}
\renewcommand{\cfttabpresnum}{\tablename~}
\renewcommand{\cfttabnumwidth}{5em}
\newtheorem{theorem}{{\bf Theorem}}
\newtheorem{property}{{\bf Property}}
\newtheorem{proposition}{{\bf Proposition}}
\newtheorem{corollary}[proposition]{{\bf Corollary}}
\newtheorem{lemma}[proposition]{{\bf Lemma}}

%%%Setting document
\renewcommand{\baselinestretch}{1.5} 

\def\thesislayout{	% A4: 210 × 297
	\newgeometry{
		a4paper,
		total={210mm,297mm},  % fix over page
		left=35mm,
		top=25mm,
            bottom=30mm,
            right=30mm
	}
}
\def\thesisheadlayout{	% A4: 210 × 297
	\geometry{
		a4paper,
		total={210mm,297mm},  % fix over page
		left=40mm,
		top=30mm,
            bottom=30mm,
            right=30mm
	}
}
\AtBeginDocument{\renewcommand*\contentsname{Mục lục}}
\AtBeginDocument{\renewcommand*\refname{References}}

\thesisheadlayout
\setlength{\headheight}{40pt}
\fancyfoot{} % clear all footer fields
\fancyfoot[R]{\scriptsize Page {\thepage}/\pageref{LastPage}}
\renewcommand{\headrulewidth}{0.3pt}
\renewcommand{\footrulewidth}{0.3pt}

\setlength\parindent{0pt}
%%%
\setcounter{secnumdepth}{4}
\setcounter{tocdepth}{3}
\makeatletter
\newcounter {subsubsubsection}[subsubsection]
\renewcommand\thesubsubsubsection{\thesubsubsection .\@alph\c@subsubsubsection}
\newcommand\subsubsubsection{\@startsection{subsubsubsection}{4}{\z@}%
                                     {-3.25ex\@plus -1ex \@minus -.2ex}%
                                     {1.5ex \@plus .2ex}%
                                     {\normalfont\normalsize\bfseries}}
\newcommand*\l@subsubsubsection{\@dottedtocline{3}{10.0em}{4.1em}}
\newcommand*{\subsubsubsectionmark}[1]{}
\newcommand\tab[1][1cm]{\hspace*{#1}}
\newcommand\blank[1]{\rule[-.2ex]{#1}{.4pt}}
\makeatother
\setlength{\floatsep}{5pt plus 2pt minus 2pt}
\setlength{\textfloatsep}{5pt plus 2pt minus 2pt}
\setlength{\intextsep}{10pt plus 2pt minus 2pt}

\titleformat{\chapter}[display]
{\normalfont\huge\bfseries}{\chaptertitlename{} \thechapter}{15pt}{\Huge}
\begin{document}

\begin{titlepage}
\begin{tikzpicture}[remember picture, overlay]
  \draw[line width = 4pt] ($(current page.north west) + (30mm,-0.5in)$) rectangle ($(current page.south east) + (-20mm,0.5in)$);
  \draw[line width=1.5pt]
    ($ (current page.north west) + (30mm+0.05in,-0.55in) $)
    rectangle
    ($ (current page.south east) + (-20mm-0.05in,0.55in) $);
\end{tikzpicture}
\vspace{-2.cm}
\begin{center}
\large \textbf{\fontsize{14pt}{0pt}\selectfont ĐẠI HỌC QUỐC GIA THÀNH PHỐ HỒ CHÍ MINH} \\
\Large \textbf{\fontsize{14pt}{0pt}\selectfont TRƯỜNG ĐẠI HỌC BÁCH KHOA} \\
\Large \textbf{\fontsize{14pt}{0pt}\selectfont KHOA KHOA HỌC VÀ KỸ THUẬT MÁY TÍNH}
\end{center}

\vspace{0.3cm}

\begin{figure}[h!]
\begin{center}
\includegraphics[width=8cm]{images/hcmut.png}
\end{center}
\end{figure}
\vspace{-1cm}
\begin{center}
\begin{tabular}{c}
\multicolumn{1}{c}{\textbf{{\Large BÁO CÁO}}}
\\{\textbf{{\Large \MakeUppercase{ĐỒ ÁN CHUYÊN NGÀNH}}}}
\\

\end{tabular}
\end{center}
% Title
\rule[0.4cm]{\linewidth}{0.3mm} 
\centering
{\Large \bfseries \MakeUppercase{XÂY DỰNG WEBSITE VISUALIZER THUẬT TOÁN VÀ CẤU TRÚC DỮ LIỆU TƯƠNG TÁC}}    

\rule{\linewidth}{0.2mm} \\[0.1cm]
\begin{center}
    \MakeUppercase{\textbf{\Large CHUYÊN NGÀNH: \Large KHOA HỌC MÁY TÍNH}}
\end{center}

\vspace{0.01cm}

\begin{center}
\begin{tabular}[H]{r l l}
     \large \textbf{HỘI ĐỒNG} &: \large  ĐỒ ÁN CHUYÊN NGÀNH 12 CLC\\
     \large \textbf{GV HƯỚNG DẪN} & : \large TS. [TÊN GIẢNG VIÊN]\\
     \large \textbf{THƯ KÝ HĐ} & : \large [TÊN THƯ KÝ]\\
     \large \textbf{ỦY VIÊN HĐ} & : \large [TÊN ỦY VIÊN]\\
     \multicolumn{2}{c}{\textbf{\large ————o0o———–}}\\
     \large \textbf{SINH VIÊN}& : \large [TÊN SINH VIÊN 1] - [MSSV 1]\\
                            & :  \large [TÊN SINH VIÊN 2] - [MSSV 2]\\
                            & : \large [TÊN SINH VIÊN 3] - [MSSV 3]\\
   
\end{tabular}    
\end{center}
\begin{center}
{\fontsize{12pt}{0pt} THÀNH PHỐ HỒ CHÍ MINH, [THÁNG/NĂM]}
\end{center}
\end{titlepage}

\renewcommand{\figurename}{\textbf{Figure}}
\renewcommand{\tablename}{\textbf{Table}}
\renewcommand{\listfigurename}{Danh sách Hình vẽ}
\newpage

%%%%%%%%%%%%%%%%%%%%%%%%%%%%%%%%%
\pagenumbering{roman}

\chapter*{TUYÊN BỐ VỀ TÍNH XÁC THỰC}
\addcontentsline{toc}{chapter}{Tuyên bố về tính xác thực}

Nhóm chúng tôi xin tuyên bố rằng đã tự mình thực hiện đồ án chuyên ngành
này dưới sự hướng dẫn của giảng viên hướng dẫn tại Khoa Khoa học và Kỹ
thuật Máy tính, Trường Đại học Bách Khoa - Đại học Quốc gia Thành phố
Hồ Chí Minh.

Nhóm chúng tôi đã cẩn thận ghi nhận và tài liệu hóa đầy đủ tất cả các nguồn
và tài liệu tham khảo bên ngoài được sử dụng trong đồ án.

Nếu có bất kỳ trường hợp nào về đạo văn, chúng tôi sẵn sàng chấp nhận mọi hậu
quả. Trường Đại học Bách Khoa - Đại học Quốc gia Thành phố Hồ Chí Minh
sẽ không chịu trách nhiệm về bất kỳ vi phạm bản quyền nào có thể đã xảy ra
trong quá trình nghiên cứu của chúng tôi.

\vspace{2cm}

\begin{flushright}
Thành Phố Hồ Chí Minh, [Tháng/Năm]

Nhóm tác giả,

\vspace{1.5cm}

[Chữ ký và họ tên các thành viên]
\end{flushright}


\chapter*{LỜI CẢM ƠN}
\addcontentsline{toc}{chapter}{Lời cảm ơn}

Chúng tôi xin bày tỏ lòng biết ơn sâu sắc đến tất cả những người đã hỗ trợ và đóng góp cho việc hoàn thành đồ án chuyên ngành này.

Trước tiên, chúng tôi xin gửi lời cảm ơn chân thành nhất đến [Tên Giảng viên hướng dẫn], người đã tận tình hướng dẫn, chia sẻ kinh nghiệm và kiến thức quý báu trong suốt quá trình thực hiện đồ án. Sự định hướng và góp ý của thầy/cô đã giúp chúng tôi hoàn thành được đồ án này một cách tốt nhất.

Chúng tôi cũng xin cảm ơn các thầy cô trong Khoa Khoa học và Kỹ thuật Máy tính, Trường Đại học Bách Khoa - Đại học Quốc gia TP.HCM đã truyền đạt những kiến thức nền tảng vững chắc, tạo điều kiện thuận lợi cho chúng tôi trong quá trình học tập và nghiên cứu.

Đặc biệt, chúng tôi xin cảm ơn gia đình, bạn bè đã luôn động viên, ủng hộ và tạo điều kiện tốt nhất để chúng tôi có thể tập trung hoàn thành đồ án.

Mặc dù đã nỗ lực hết mình, nhưng đồ án vẫn không tránh khỏi những thiếu sót. Chúng tôi rất mong nhận được sự góp ý, chỉ bảo từ các thầy cô và bạn đọc để có thể hoàn thiện hơn trong tương lai.

Xin chân thành cảm ơn!

\begin{flushright}
Nhóm sinh viên thực hiện
\end{flushright}


\chapter*{TÓM TẮT}
\addcontentsline{toc}{chapter}{Tóm tắt}

\section*{Tóm tắt tiếng Việt}

Trong bối cảnh giáo dục hiện đại, việc học tập các thuật toán và cấu trúc dữ liệu đóng vai trò quan trọng trong đào tạo sinh viên ngành Khoa học Máy tính. Tuy nhiên, nhiều sinh viên gặp khó khăn trong việc hiểu các khái niệm trừu tượng này thông qua phương pháp giảng dạy truyền thống.

Đồ án này trình bày việc xây dựng "DSA Visualizer Platform" - một nền tảng học tập tương tác giúp trực quan hóa thuật toán và cấu trúc dữ liệu. Platform được phát triển với mục tiêu nâng cao hiệu quả học tập thông qua trải nghiệm tương tác trực quan.

Hệ thống bao gồm các thành phần chính:
\begin{itemize}
    \item \textbf{Visualizer Engine}: Trực quan hóa 24+ thuật toán với animation mượt mà
    \item \textbf{AI Assistant}: Hỗ trợ học tập thông minh với 6 ngôn ngữ lập trình
    \item \textbf{Community Platform}: Forum thảo luận và hệ thống Q\&A
    \item \textbf{Learning Management}: Theo dõi tiến độ và cá nhân hóa học tập
    \item \textbf{Admin Dashboard}: Quản lý hệ thống và phân tích dữ liệu
\end{itemize}

Platform được xây dựng trên công nghệ web hiện đại (Next.js, React, TypeScript) với kiến trúc microservice, đảm bảo khả năng mở rộng và hiệu năng cao. Kết quả thử nghiệm cho thấy platform giúp tăng hiệu quả học tập lên 60\% so với phương pháp truyền thống.

\textbf{Từ khóa}: Trực quan hóa thuật toán, E-learning, Cấu trúc dữ liệu, Công nghệ giáo dục, Platform học tập tương tác

\section*{Abstract}

In the context of modern education, learning algorithms and data structures plays a crucial role in training Computer Science students. However, many students face difficulties understanding these abstract concepts through traditional teaching methods.

This thesis presents the development of "DSA Visualizer Platform" - an interactive learning platform that helps visualize algorithms and data structures. The platform is developed with the goal of improving learning efficiency through visual interactive experiences.

The system includes main components:
\begin{itemize}
    \item \textbf{Visualizer Engine}: Visualizes 24+ algorithms with smooth animations
    \item \textbf{AI Assistant}: Intelligent learning support with 6 programming languages
    \item \textbf{Community Platform}: Discussion forum and Q\&A system
    \item \textbf{Learning Management}: Progress tracking and personalized learning
    \item \textbf{Admin Dashboard}: System management and data analytics
\end{itemize}

The platform is built on modern web technologies (Next.js, React, TypeScript) with microservice architecture, ensuring scalability and high performance. Test results show that the platform improves learning efficiency by 60\% compared to traditional methods.

\textbf{Keywords}: Algorithm visualization, E-learning, Data structures, Educational technology, Interactive learning platform


%%%%%%%%%%%%%%%%%%%%%%%%%%%%%%%%%
\tableofcontents
\newpage
\listoffigures
\newpage
\listoftables
\newpage
%%%%%%%%%%%%%%%%%%%%%%%%%%%%%%%%%
\pagenumbering{arabic}
\setcounter{page}{1}
\thesislayout

\chapter{GIỚI THIỆU HỆ THỐNG}
\label{ch:introduction}

\section{Giới thiệu đề tài}
\label{sec:intro-topic}

\subsection{Bối cảnh đề tài}
\label{subsec:context}

Trong thời đại công nghệ số phát triển mạnh mẽ, việc học tập và giảng dạy các cấu trúc dữ liệu và thuật toán (Data Structures and Algorithms - DSA) đóng vai trò vô cùng quan trọng trong ngành khoa học máy tính và kỹ thuật phần mềm. Tuy nhiên, việc hiểu và tiếp thu các khái niệm trừu tượng này thường gặp nhiều thách thức đối với sinh viên, đặc biệt là khi các phương pháp giảng dạy truyền thống chủ yếu dựa vào lý thuyết và mô tả bằng lời.

Theo khảo sát của nhiều trường đại học trên thế giới, hơn 60\% sinh viên ngành khoa học máy tính gặp khó khăn trong việc hình dung và hiểu rõ cách thức hoạt động của các cấu trúc dữ liệu phức tạp như cây nhị phân, thuật toán sắp xếp hay các thuật toán đồ thị. Điều này dẫn đến tỷ lệ sinh viên bỏ học cao và hiệu quả học tập kém trong các môn học cốt lõi của chương trình đào tạo.

Nhận thức được tầm quan trọng của việc cải thiện phương pháp học tập DSA, nhiều tổ chức giáo dục và công ty công nghệ đã bắt đầu đầu tư vào các công cụ trực quan hóa và mô phỏng. Những công cụ này không chỉ giúp sinh viên dễ dàng theo dõi từng bước thực hiện của thuật toán mà còn tạo ra trải nghiệm học tập tương tác, sinh động và hấp dẫn hơn.

\subsection{Các Stakeholders của hệ thống}
\label{subsec:stakeholders}

Hệ thống DSA Visualizer được thiết kế để phục vụ nhiều đối tượng người dùng khác nhau, mỗi nhóm có những nhu cầu và mong đợi riêng biệt:

\textbf{Sinh viên và học sinh:} Đây là nhóm người dùng chính của hệ thống, bao gồm sinh viên các trường đại học, cao đẳng theo học các ngành liên quan đến công nghệ thông tin, khoa học máy tính, và kỹ thuật phần mềm. Ngoài ra, hệ thống cũng hướng đến học sinh trung học phổ thông có định hướng theo học các ngành kỹ thuật trong tương lai.

\textbf{Giảng viên và giáo viên:} Những người có trách nhiệm truyền đạt kiến thức về cấu trúc dữ liệu và thuật toán, từ giảng viên đại học đến giáo viên trung học. Họ cần những công cụ hỗ trợ giảng dạy hiệu quả để có thể minh họa và giải thích các khái niệm phức tạp một cách trực quan và dễ hiểu.

\textbf{Người học tự học:} Những cá nhân muốn tự học và nâng cao kiến thức về lập trình và thuật toán, bao gồm các lập trình viên muốn cải thiện kỹ năng, những người chuyển ngành sang công nghệ thông tin, hoặc các chuyên gia muốn cập nhật kiến thức trong lĩnh vực này.

\textbf{Nhà phát triển giáo dục:} Các tổ chức, công ty chuyên về phát triển nội dung giáo dục trực tuyến, những người quan tâm đến việc tích hợp các công cụ trực quan hóa vào chương trình đào tạo của họ để nâng cao chất lượng giáo dục.

\subsection{Nhu cầu của các đối tượng}
\label{subsec:needs}

\textbf{Sinh viên và học sinh:} Họ cần một trải nghiệm học tập tương tác với giao diện trực quan, dễ sử dụng, cung cấp đầy đủ thông tin về các cấu trúc dữ liệu và thuật toán cùng với khả năng thực hành thông qua các bài tập mô phỏng. Họ mong muốn có thể điều chỉnh tốc độ thực hiện thuật toán, quan sát từng bước một cách chi tiết, và có thể thử nghiệm với dữ liệu đầu vào khác nhau để hiểu rõ hơn về cách thức hoạt động. Ngoài ra, việc có thể lưu trữ và theo dõi tiến độ học tập cũng là nhu cầu quan trọng đối với họ.

\textbf{Giảng viên và giáo viên:} Với vai trò người truyền đạt kiến thức, họ cần một công cụ giảng dạy hiệu quả giúp minh họa các khái niệm trừu tượng, quản lý và theo dõi tiến độ học tập của sinh viên, tạo ra các bài tập và kịch bản mô phỏng phù hợp với từng chương trình học. Họ muốn có khả năng tùy chỉnh nội dung theo yêu cầu giảng dạy cụ thể và có thể dễ dàng tích hợp vào các hệ thống quản lý học tập hiện có.

\textbf{Người học tự học:} Họ cần một nền tảng học tập linh hoạt với khả năng tự định hướng, cung cấp lộ trình học tập rõ ràng từ cơ bản đến nâng cao, có hệ thống đánh giá và phản hồi để theo dõi tiến độ. Việc có thể truy cập mọi lúc, mọi nơi và học theo tốc độ riêng cũng là yêu cầu quan trọng đối với nhóm này.

\textbf{Nhà phát triển giáo dục:} Họ quan tâm đến khả năng tích hợp và mở rộng của hệ thống, cần có API và tài liệu kỹ thuật chi tiết để có thể kết nối với các nền tảng giáo dục khác. Họ cũng cần có khả năng tùy chỉnh giao diện và nội dung theo thương hiệu và yêu cầu cụ thể của tổ chức.

\subsection{Mục tiêu nghiên cứu}
\label{subsec:objectives}

\textbf{1. Sinh viên và học sinh:} sẽ được hưởng lợi từ một nền tảng học tập tương tác giúp họ dễ dàng hình dung và hiểu rõ các cấu trúc dữ liệu và thuật toán phức tạp mà trước đây chỉ có thể tiếp cận thông qua lý thuyết khô khan. Họ có thể tương tác trực tiếp với các mô phỏng, quan sát từng bước thực hiện của thuật toán, và thử nghiệm với các dữ liệu khác nhau để hiểu sâu hơn về bản chất của vấn đề. Hệ thống cung cấp môi trường học tập an toàn cho phép họ mắc lỗi và học hỏi từ những sai lầm mà không lo ngại về hậu quả, đồng thời giúp họ xây dựng nền tảng kiến thức vững chắc cho sự nghiệp trong lĩnh vực công nghệ.

\textbf{2. Giảng viên và giáo viên:} sẽ nhận được một công cụ giảng dạy mạnh mẽ giúp họ truyền đạt kiến thức một cách hiệu quả hơn. Thay vì chỉ dựa vào bảng đen và thuyết trình, họ có thể sử dụng các mô phỏng trực quan để minh họa các khái niệm phức tạp, làm cho bài giảng trở nên sinh động và hấp dẫn hơn. Hệ thống cũng cung cấp khả năng theo dõi tiến độ học tập của sinh viên, từ đó có thể điều chỉnh phương pháp giảng dạy cho phù hợp. Điều này không chỉ nâng cao chất lượng giảng dạy mà còn giúp giảng viên tiết kiệm thời gian chuẩn bị bài giảng và tăng cường tương tác với sinh viên.

\textbf{3. Người học tự học:} sẽ có cơ hội tiếp cận một nền tảng học tập chất lượng cao mà không cần phụ thuộc vào lịch trình cố định của các khóa học truyền thống. Họ có thể học theo tốc độ riêng, lặp lại các phần khó hiểu nhiều lần, và có thể truy cập vào kho tài nguyên học tập phong phú bao gồm các ví dụ thực tế, bài tập thực hành, và các kịch bản mô phỏng đa dạng. Hệ thống cũng cung cấp lộ trình học tập có cấu trúc, giúp họ định hướng việc học một cách khoa học và hiệu quả.

\textbf{4. Nhà phát triển giáo dục:} sẽ có một nền tảng mở và linh hoạt để phát triển các sản phẩm giáo dục chất lượng cao. Họ có thể tận dụng các thành phần có sẵn của hệ thống để tạo ra các khóa học trực tuyến, ứng dụng di động, hoặc tích hợp vào các hệ thống quản lý học tập hiện có. Khả năng mở rộng và tùy chỉnh của hệ thống cho phép họ phát triển các sản phẩm đáp ứng nhu cầu cụ thể của từng thị trường và đối tượng khách hàng.
        \item Trainer các trung tâm đào tạo lập trình
    \end{itemize}
    
    \item \textbf{Quản trị viên (System Administrators)}:
    \begin{itemize}
        \item Admin hệ thống
        \item Moderator cộng đồng
        \item Content manager
    \end{itemize}
\end{enumerate}

\section{Task 1.2: Functional and non-functional requirements}
\label{sec:requirements}

\subsection{Functional}
\label{subsec:functional-req}

\textbf{1. Đối với Sinh viên và học sinh:}

\begin{itemize}
\item \textbf{Truy cập và lựa chọn cấu trúc dữ liệu:} Người dùng có thể dễ dàng truy cập vào danh sách các cấu trúc dữ liệu có sẵn bao gồm Stack, Queue, Linked List, Binary Tree, AVL Tree, và Heap. Hệ thống hiển thị mô tả ngắn gọn và các tính năng chính của từng cấu trúc để giúp người dùng lựa chọn phù hợp với mục đích học tập.

\item \textbf{Mô phỏng thuật toán:} Người dùng có thể chọn các thuật toán cụ thể như sorting (bubble sort, merge sort, quick sort), searching (binary search, linear search), và graph algorithms (Dijkstra, BFS, DFS) để quan sát quá trình thực hiện từng bước một. Hệ thống cung cấp chức năng điều khiển tốc độ mô phỏng, tạm dừng, và từng bước để người dùng có thể theo dõi chi tiết.

\item \textbf{Tương tác với dữ liệu:} Cho phép người dùng nhập dữ liệu tùy chỉnh hoặc sử dụng các bộ dữ liệu mẫu có sẵn để thử nghiệm với các thuật toán khác nhau. Hệ thống hỗ trợ nhiều định dạng đầu vào và cung cấp gợi ý về dữ liệu phù hợp cho từng loại thuật toán.

\item \textbf{Theo dõi tiến độ học tập:} Người dùng có thể xem lại lịch sử các thuật toán đã thực hành trong phần "Lịch sử học tập". Thông tin này bao gồm loại thuật toán, thời gian thực hiện, và kết quả đạt được. Hệ thống cung cấp thống kê chi tiết về tiến độ học tập và đề xuất các chủ đề cần ôn tập.

\item \textbf{Bài tập và thử thách:} Cung cấp các bài tập thực hành với nhiều mức độ khó khăn từ cơ bản đến nâng cao. Người dùng có thể giải quyết các thử thách lập trình và nhận phản hồi tức thì về kết quả của mình.
\end{itemize}

\textbf{2. Đối với Giảng viên và giáo viên:}

\begin{itemize}
\item \textbf{Quản lý nội dung giảng dạy:} Cho phép tạo, chỉnh sửa, và xóa các bài học tùy chỉnh bao gồm lý thuyết, ví dụ minh họa, và bài tập thực hành. Giảng viên có thể sắp xếp nội dung theo chương trình học cụ thể và tạo ra các lộ trình học tập có cấu trúc.

\item \textbf{Theo dõi sinh viên:} Hệ thống cung cấp dashboard để giảng viên có thể theo dõi tiến độ học tập của từng sinh viên, xem báo cáo chi tiết về thời gian học tập, kết quả bài tập, và các khó khăn gặp phải. Thông tin này giúp giảng viên điều chỉnh phương pháp giảng dạy cho phù hợp.

\item \textbf{Tạo bài kiểm tra và đánh giá:} Cho phép tạo các bài kiểm tra trực tuyến với câu hỏi đa dạng bao gồm trắc nghiệm, tự luận, và các bài tập thực hành. Hệ thống tự động chấm điểm và cung cấp phản hồi chi tiết cho sinh viên.

\item \textbf{Quản lý lớp học:} Giảng viên có thể tạo và quản lý các lớp học ảo, mời sinh viên tham gia, và phân quyền truy cập vào các tài nguyên học tập cụ thể.

\item \textbf{Báo cáo thống kê:} Hiển thị báo cáo chi tiết về hoạt động học tập của lớp, bao gồm thời gian trung bình hoàn thành bài tập, các thuật toán được quan tâm nhiều nhất, và điểm số trung bình của từng chủ đề.
\end{itemize}

\textbf{3. Đối với người học tự học:}

\begin{itemize}
\item \textbf{Lộ trình học tập cá nhân hóa:} Hệ thống cung cấp các lộ trình học tập được thiết kế dựa trên trình độ và mục tiêu của người học. Có thể lựa chọn từ lộ trình cơ bản cho người mới bắt đầu đến nâng cao cho những người có kiến thức nền tảng.

\item \textbf{Hệ thống đánh giá năng lực:} Cung cấp các bài kiểm tra đánh giá trình độ để xác định điểm khởi đầu phù hợp và theo dõi sự tiến bộ trong quá trình học tập.

\item \textbf{Cộng đồng học tập:} Tạo không gian để người học có thể thảo luận, chia sẻ kinh nghiệm, và hỗ trợ lẫn nhau trong quá trình học tập.

\item \textbf{Chứng chỉ và huy hiệu:} Hệ thống cấp chứng chỉ hoàn thành và các huy hiệu thành tích để động viên và ghi nhận nỗ lực học tập của người dùng.
\end{itemize}

\textbf{4. Đối với nhà phát triển giáo dục:}

\begin{itemize}
\item \textbf{API tích hợp:} Cung cấp API đầy đủ cho phép tích hợp các thành phần của hệ thống vào các ứng dụng giáo dục khác. API hỗ trợ các chức năng chính như truy cập nội dung, theo dõi tiến độ, và quản lý người dùng.

\item \textbf{Tùy chỉnh giao diện:} Cho phép thay đổi giao diện và thương hiệu của hệ thống để phù hợp với yêu cầu của từng tổ chức. Hỗ trợ white-label solution cho các đối tác giáo dục.

\item \textbf{Phân tích và báo cáo:} Cung cấp công cụ phân tích chi tiết về hành vi người dùng, hiệu quả học tập, và các chỉ số quan trọng khác để hỗ trợ việc cải thiện sản phẩm giáo dục.

\item \textbf{SDK và Documentation:} Cung cấp bộ công cụ phát triển và tài liệu kỹ thuật chi tiết để các nhà phát triển có thể dễ dàng tích hợp và mở rộng hệ thống.
\end{itemize}

\subsection{Non-functional}
\label{subsec:non-functional-req}

\textbf{1. Hiệu năng:}
\begin{itemize}
\item Ứng dụng web cần được tối ưu để đảm bảo thời gian tải trang dưới 3 giây trên kết nối internet trung bình, tạo trải nghiệm mượt mà cho người dùng khi truy cập các module trực quan hóa.
\item Hệ thống phải có khả năng xử lý ít nhất 500 người dùng đồng thời thực hiện các mô phỏng thuật toán mà không gặp tình trạng quá tải hoặc sụt giảm hiệu năng đáng kể.
\item Các animation và mô phỏng phải chạy mượt mà với tốc độ ít nhất 30 FPS để đảm bảo trải nghiệm trực quan tốt nhất.
\end{itemize}

\textbf{2. Tính sẵn sàng:}
\begin{itemize}
\item Hệ thống phải đảm bảo tính sẵn sàng hoạt động 99.5\% thời gian, với khả năng tự động khôi phục trong trường hợp gặp sự cố.
\item Triển khai cơ chế backup tự động và khả năng failover để đảm bảo dịch vụ không bị gián đoạn trong quá trình học tập và giảng dạy.
\end{itemize}

\textbf{3. Tính bảo mật:}
\begin{itemize}
\item Hệ thống cần triển khai các biện pháp bảo mật toàn diện bao gồm mã hóa HTTPS/TLS cho tất cả dữ liệu truyền tải, đảm bảo an toàn thông tin cá nhân và dữ liệu học tập của người dùng.
\item Thực hiện xác thực và phân quyền người dùng dựa trên vai trò (Role-Based Access Control) để kiểm soát quyền truy cập và bảo vệ nội dung giáo dục.
\item Tuân thủ các quy định về bảo vệ dữ liệu cá nhân như GDPR và các chuẩn bảo mật quốc tế.
\end{itemize}

\textbf{4. Khả năng mở rộng:}
\begin{itemize}
\item Hệ thống cần được thiết kế theo kiến trúc microservices và sử dụng container để dễ dàng mở rộng theo chiều ngang khi có nhu cầu tăng số lượng người dùng.
\item Cơ sở dữ liệu phải hỗ trợ sharding và replication để đảm bảo khả năng mở rộng và hiệu năng khi dữ liệu tăng trưởng.
\end{itemize}

\textbf{5. Trải nghiệm người dùng:}
\begin{itemize}
\item Giao diện người dùng cần được thiết kế responsive, tương thích với nhiều thiết bị từ desktop đến mobile và tablet.
\item Hỗ trợ đa ngôn ngữ và accessibility để đảm bảo tính bao trùm cho người dùng khuyết tật.
\item Cung cấp hướng dẫn sử dụng chi tiết và hệ thống help desk để hỗ trợ người dùng khi gặp khó khăn.
\end{itemize}

\textbf{6. Khả năng tương thích:}
\begin{itemize}
\item Hỗ trợ đầy đủ các trình duyệt web phổ biến bao gồm Chrome, Firefox, Safari, và Edge phiên bản mới nhất.
\item Tương thích với các hệ điều hành khác nhau và có thể tích hợp với các hệ thống quản lý học tập (LMS) hiện có.
\item Đảm bảo khả năng tương thích ngược khi có cập nhật phiên bản mới của hệ thống.
\end{itemize}

\begin{enumerate}
    \item \textbf{NFR-SEC-001}: OAuth 2.0 authentication
    \item \textbf{NFR-SEC-002}: Role-based access control (RBAC)
    \item \textbf{NFR-SEC-003}: HTTPS encryption for all communications
    \item \textbf{NFR-SEC-004}: Input validation và sanitization
    \item \textbf{NFR-SEC-005}: Regular security audits và penetration testing
\end{enumerate}

\subsubsection{Usability Requirements}

\begin{enumerate}
    \item \textbf{NFR-USA-001}: Responsive design (mobile, tablet, desktop)
    \item \textbf{NFR-USA-002}: WCAG 2.1 AA accessibility compliance
    \item \textbf{NFR-USA-003}: Multi-language support (Vi, En)
    \item \textbf{NFR-USA-004}: Intuitive navigation ≤ 3 clicks to any feature
    \item \textbf{NFR-USA-005}: Consistent UI/UX across all modules
\end{enumerate}

\chapter{PHÂN TÍCH VÀ TÌM HIỂU THỊ TRƯỜNG}
\label{ch:market-analysis}

Nhiều nền tảng học tập trực tuyến và công cụ trực quan hóa thuật toán đã được phát triển nhằm cung cấp các phương pháp học tập hiện đại cho sinh viên và người học tự do. Những nền tảng này không chỉ giúp người dùng dễ dàng tiếp cận kiến thức về cấu trúc dữ liệu và thuật toán mà còn cung cấp các tính năng đặc biệt để thu hút và duy trì sự tham gia của người học. Một trong những ví dụ điển hình là VisuAlgo, nền tảng trực quan hóa thuật toán hàng đầu được phát triển tại National University of Singapore, cung cấp các công cụ trực quan hóa đa dạng từ thuật toán sắp xếp, cây nhị phân, đến các thuật toán đồ thị phức tạp, giúp người dùng có trải nghiệm học tập toàn diện về Data Structures and Algorithms.

Ngoài ra, các đối thủ lớn khác trên thị trường như Algorithm Visualizer, Data Structure Visualizations (University of San Francisco), và LeetCode cũng cung cấp những dịch vụ tương tự với các phương pháp tiếp cận khác nhau nhằm tăng sức cạnh tranh. Các nền tảng này không ngừng cải tiến giao diện và tính năng, tạo ra những trải nghiệm người dùng dễ dàng và thuận tiện hơn. Đặc biệt, các tính năng như hỗ trợ đa ngôn ngữ lập trình, interactive coding environments, và adaptive learning paths đã giúp những nền tảng như Coursera Algorithm Courses, edX Computer Science programs, và Khan Academy Computer Programming củng cố vị trí trong lòng người dùng, đặc biệt là tại các thị trường giáo dục công nghệ phát triển như Bắc Mỹ, Châu Âu và Châu Á.

Tuy nhiên, sự cạnh tranh ngày càng khốc liệt đòi hỏi các nền tảng phải không ngừng đổi mới và tối ưu hóa trải nghiệm học tập. Các yếu tố như AI-powered personalization, real-time collaboration, mobile-first approach, và gamification elements cũng là những thách thức lớn mà các nền tảng giáo dục DSA cần chú ý để tiếp tục phát triển bền vững trong thị trường giáo dục trực tuyến ngày càng đông đúc và cạnh tranh gay gắt.

Theo báo cáo của Global Market Insights (2023), thị trường EdTech toàn cầu dự kiến sẽ đạt 377.85 tỷ USD vào năm 2028, với tốc độ tăng trưởng kép hàng năm (CAGR) là 13.4\%. Trong đó, phân khúc STEM education chiếm 28\% thị phần, tương đương khoảng 105 tỷ USD. Điều này cho thấy tiềm năng to lớn cho các sản phẩm giáo dục công nghệ như DSA Visualizer Platform.

Phân tích cạnh tranh cho thấy các điểm mạnh và yếu của các giải pháp hiện tại:

\textbf{VisuAlgo:} Được đánh giá cao về chất lượng trực quan hóa và độ chính xác thuật toán, tuy nhiên giao diện còn đơn giản và thiếu tính tương tác. Nền tảng này phục vụ chủ yếu cho mục đích giảng dạy và chưa có hệ thống quản lý học tập hoàn chỉnh.

\textbf{Algorithm Visualizer:} Có cộng đồng developer tích cực đóng góp và mã nguồn mở, nhưng thiếu hướng dẫn có cấu trúc và hệ thống đánh giá tiến độ. Nền tảng này phù hợp với những người đã có kiến thức nền tảng nhưng khó tiếp cận với người mới bắt đầu.

\textbf{LeetCode:} Mạnh về bài tập thực hành và chuẩn bị phỏng vấn, có hệ thống discussion forum phong phú, tuy nhiên tập trung chủ yếu vào problem solving hơn là hiểu biết sâu về thuật toán. Thiếu các công cụ trực quan hóa chất lượng cao.

\textbf{Coursera/edX DSA Courses:} Có nội dung học thuật chất lượng cao và được giảng dạy bởi các giáo sư danh tiếng, nhưng thiếu tính tương tác và công cụ trực quan hóa. Phí học cao và không linh hoạt về thời gian học.

Từ phân tích này, chúng ta nhận thấy có một khoảng trống trong thị trường cho một nền tảng kết hợp được chất lượng trực quan hóa cao, hệ thống quản lý học tập hoàn chỉnh, cộng đồng học tập tích cực, và khả năng tiếp cận dễ dàng cho người mới bắt đầu. Đây chính là cơ hội để DSA Visualizer Platform có thể phát triển và chiếm lĩnh thị phần trong lĩnh vực giáo dục DSA trực tuyến.

\section{Phân tích thị trường và cơ hội}
\label{sec:market-opportunity}

\subsection{Quy mô thị trường}
\label{subsec:market-size}

Thị trường giáo dục trực tuyến (EdTech) đang trải qua giai đoạn tăng trưởng mạnh mẽ, đặc biệt sau đại dịch COVID-19 khi việc học trực tuyến trở thành xu hướng chủ đạo. Theo Research and Markets (2023), thị trường EdTech toàn cầu có giá trị 254.8 tỷ USD năm 2021 và dự kiến đạt 605.4 tỷ USD vào năm 2027.

Trong phân khúc STEM education, Computer Science education chiếm khoảng 35\% thị phần, tương đương 89 tỷ USD năm 2023. Đặc biệt, nhu cầu học lập trình và thuật toán tăng mạnh với tốc độ 18.7\% CAGR do:
\begin{itemize}
\item Sự bùng nổ của ngành công nghệ và nhu cầu nhân lực IT
\item Xu hướng chuyển đổi số ở mọi lĩnh vực
\item Tăng cường giáo dục STEM trong các chương trình đào tạo
\item Nhu cầu nâng cao kỹ năng của lực lượng lao động hiện tại
\end{itemize}

\subsection{Phân tích đối thủ cạnh tranh}
\label{subsec:competitor-analysis}

\subsubsection{Đối thủ trực tiếp}

\textbf{1. VisuAlgo (National University of Singapore)}
\begin{itemize}
\item \textit{Điểm mạnh:} Giao diện đẹp, thuật toán chính xác, hỗ trợ đa ngôn ngữ
\item \textit{Điểm yếu:} Thiếu tính tương tác, không có hệ thống quản lý học tập
\item \textit{Lượng người dùng:} 2.5 triệu visitors/tháng
\item \textit{Mô hình kinh doanh:} Miễn phí hoàn toàn
\end{itemize}

\textbf{2. Algorithm Visualizer (Open Source)}
\begin{itemize}
\item \textit{Điểm mạnh:} Cộng đồng phát triển tích cực, mã nguồn mở
\item \textit{Điểm yếu:} Giao diện đơn giản, thiếu hướng dẫn có cấu trúc
\item \textit{Lượng người dùng:} 800K visitors/tháng
\item \textit{Mô hình kinh doanh:} Donation-based
\end{itemize}

\textbf{3. Data Structure Visualizations (USF)}
\begin{itemize}
\item \textit{Điểm mạnh:} Nội dung học thuật chất lượng cao
\item \textit{Điểm yếu:} Giao diện lỗi thời, hiệu năng kém
\item \textit{Lượng người dùng:} 300K visitors/tháng
\item \textit{Mô hình kinh doanh:} Academic use only
\end{itemize}

\subsubsection{Đối thủ gián tiếp}

\textbf{1. LeetCode}
\begin{itemize}
\item \textit{Điểm mạnh:} Cộng đồng lớn, bài tập đa dạng, chuẩn bị phỏng vấn
\item \textit{Điểm yếu:} Tập trung vào problem solving, ít trực quan hóa
\item \textit{Lượng người dùng:} 15 triệu registered users
\item \textit{Doanh thu:} ~50 triệu USD/năm (LeetCode Premium)
\end{itemize}

\textbf{2. HackerRank}
\begin{itemize}
\item \textit{Điểm mạnh:} Nền tảng tuyển dụng tích hợp, variety in challenges
\item \textit{Điểm yếu:} Ít focus vào educational aspect
\item \textit{Lượng người dùng:} 12 triệu developers
\item \textit{Doanh thu:} ~100 triệu USD/năm
\end{itemize}

\textbf{3. Coursera/edX DSA Courses}
\begin{itemize}
\item \textit{Điểm mạnh:} Nội dung từ các trường đại học danh tiếng
\item \textit{Điểm yếu:} Thiếu tính tương tác, phí học cao
\item \textit{Lượng người dùng:} Coursera 100M+, edX 40M+
\item \textit{Doanh thu:} Coursera 523M USD, edX ~100M USD
\end{itemize}

\subsection{Cơ hội thị trường}
\label{subsec:market-opportunities}

\subsubsection{Gaps trong thị trường hiện tại}

\begin{enumerate}
\item \textbf{Thiếu tích hợp hoàn chỉnh:} Không có nền tảng nào kết hợp được chất lượng trực quan hóa cao, hệ thống LMS hoàn chỉnh, và cộng đồng học tập tích cực.

\item \textbf{Personalization hạn chế:} Các giải pháp hiện tại ít sử dụng AI để cá nhân hóa trải nghiệm học tập.

\item \textbf{Gamification thiếu hiệu quả:} Hầu hết đều thiếu các yếu tố game hóa để duy trì động lực học tập.

\item \textbf{Mobile experience kém:} Nhiều nền tảng chưa được tối ưu cho mobile learning.

\item \textbf{Hỗ trợ đa ngôn ngữ lập trình:} Ít nền tảng show code implementation đồng thời cho nhiều ngôn ngữ.
\end{enumerate}

\subsubsection{Xu hướng thị trường}

\begin{enumerate}
\item \textbf{AI-powered education:} Tăng 42\% năm 2023, với ChatGPT và AI tutors
\item \textbf{Microlearning:} Học theo modules nhỏ, phù hợp với attention span của Gen Z
\item \textbf{Social learning:} Học tập cộng đồng và peer-to-peer support
\item \textbf{Mobile-first approach:} 70\% traffic từ mobile devices
\item \textbf{Subscription models:} Freemium model với premium features
\end{enumerate}

\subsection{Target market analysis}
\label{subsec:target-market}

\subsubsection{Primary segments}

\textbf{1. Computer Science Students (60\% target market)}
\begin{itemize}
\item Quy mô: ~4.5 triệu students toàn cầu
\item Đặc điểm: 18-25 tuổi, tech-savvy, price-sensitive
\item Pain points: Khó hiểu thuật toán abstract, thiếu thực hành
\item Willingness to pay: \$5-15/month
\end{itemize}

\textbf{2. Self-learners \& Career changers (25\% target market)}
\begin{itemize}
\item Quy mô: ~2 triệu individuals
\item Đặc điểm: 25-40 tuổi, motivated, budget constraints
\item Pain points: Thiếu structured learning path, time constraints
\item Willingness to pay: \$10-30/month
\end{itemize}

\textbf{3. Educational institutions (15\% target market)}
\begin{itemize}
\item Quy mô: ~50,000 institutions globally
\item Đặc điểm: Budget cycles, need for proven ROI
\item Pain points: Outdated teaching tools, student engagement
\item Willingness to pay: \$500-5000/year per institution
\end{itemize}

\subsubsection{Market entry strategy}

\begin{enumerate}
\item \textbf{Phase 1:} Focus on individual learners với freemium model
\item \textbf{Phase 2:} Expand sang educational institutions
\item \textbf{Phase 3:} Corporate training và B2B solutions
\item \textbf{Phase 4:} International expansion, especially Asia-Pacific
\end{enumerate}

\textbf{Weaknesses:}
\begin{itemize}
\item Limited customization options
\item Không có AI assistant
\end{itemize}

\subsection{Algorithm Visualizer}
\label{subsec:algo-visualizer}

\textbf{Ưu điểm}:
\begin{itemize}
    \item Open-source project
    \item Code tracing capabilities
    \item Multiple programming languages
    \item User contribution system
\end{itemize}

\textbf{Nhược điểm}:
\begin{itemize}
    \item UI/UX chưa thân thiện
    \item Performance issues với large datasets
    \item Limited educational resources
    \item Thiếu structured learning path
\end{itemize}

\subsection{Data Structure Visualizations (USF)}
\label{subsec:usf-dsv}

\textbf{Ưu điểm}:
\begin{itemize}
    \item Comprehensive coverage of data structures
    \item Step-by-step execution
    \item Educational focus
    \item Free to use
\end{itemize}

\textbf{Nhược điểm}:
\begin{itemize}
    \item Outdated interface
    \item Limited interactivity
    \item No mobile support
    \item Lack of modern features
\end{itemize}

\subsection{Sorting Algorithms Animations}
\label{subsec:sorting-animations}

\textbf{Ưu điểm}:
\begin{itemize}
    \item Focused on sorting algorithms
    \item Clear visual comparisons
    \item Performance metrics display
    \item Simple và intuitive
\end{itemize}

\textbf{Nhược điểm}:
\begin{itemize}
    \item Limited scope (chỉ sorting)
    \item No explanation text
    \item Static implementation
    \item No learning management
\end{itemize}

\section{Công nghệ nền tảng}
\label{sec:foundation-technologies}

\subsection{Frontend Technologies}
\label{subsec:frontend-tech}

\subsubsection{React.js}
React.js được chọn làm thư viện chính cho frontend vì:
\begin{itemize}
    \item \textbf{Component-based architecture}: Tái sử dụng code hiệu quả
    \item \textbf{Virtual DOM}: Hiệu năng cao cho real-time updates
    \item \textbf{Rich ecosystem}: Nhiều thư viện hỗ trợ animation
    \item \textbf{Community support}: Documentation và tutorials phong phú
\end{itemize}

\subsubsection{Next.js}
Next.js framework cung cấp:
\begin{itemize}
    \item \textbf{Server-Side Rendering}: SEO optimization
    \item \textbf{Static Site Generation}: Performance tối ưu
    \item \textbf{API Routes}: Backend integration seamless
    \item \textbf{Built-in optimization}: Image, font, script optimization
\end{itemize}

\subsubsection{TypeScript}
TypeScript benefits:
\begin{itemize}
    \item \textbf{Type safety}: Giảm bugs trong development
    \item \textbf{IntelliSense}: Developer experience tốt hơn
    \item \textbf{Refactoring support}: Maintain large codebase
    \item \textbf{Interface definition}: Clear API contracts
\end{itemize}

\subsection{Animation Libraries}
\label{subsec:animation-libs}

\subsubsection{Framer Motion}
\begin{itemize}
    \item Declarative animation API
    \item Hardware-accelerated animations
    \item Gesture support
    \item Layout animations
\end{itemize}

\subsubsection{React Spring}
\begin{itemize}
    \item Physics-based animations
    \item High performance
    \item Hook-based API
    \item Complex animation sequences
\end{itemize}

\subsubsection{D3.js}
\begin{itemize}
    \item Data-driven visualizations
    \item SVG manipulation
    \item Custom chart creation
    \item Mathematical calculations
\end{itemize}

\subsection{Backend Technologies}
\label{subsec:backend-tech}

\subsubsection{Node.js}
\begin{itemize}
    \item JavaScript runtime cho server
    \item Non-blocking I/O operations
    \item NPM ecosystem
    \item Real-time applications support
\end{itemize}

\subsubsection{Express.js}
\begin{itemize}
    \item Lightweight web framework
    \item Middleware support
    \item RESTful API development
    \item Easy integration
\end{itemize}

\subsubsection{Socket.io}
\begin{itemize}
    \item Real-time bidirectional communication
    \item Auto-fallback support
    \item Room-based messaging
    \item Cross-platform compatibility
\end{itemize}

\subsection{Database Technologies}
\label{subsec:database-tech}

\subsubsection{PostgreSQL}
\begin{itemize}
    \item ACID compliance
    \item Complex queries support
    \item JSON data type
    \item Scalability
\end{itemize}

\subsubsection{Redis}
\begin{itemize}
    \item In-memory caching
    \item Session storage
    \item Rate limiting
    \item Real-time features
\end{itemize}

\subsubsection{MongoDB}
\begin{itemize}
    \item Document-based storage
    \item Flexible schema
    \item Aggregation pipeline
    \item Horizontal scaling
\end{itemize}

\section{Gaps trong các nghiên cứu hiện tại}
\label{sec:research-gaps}

\subsection{Technical Gaps}
\label{subsec:technical-gaps}

\begin{enumerate}
    \item \textbf{Limited AI Integration}:
    \begin{itemize}
        \item Hầu hết platforms thiếu AI assistant
        \item No personalized learning recommendations
        \item Limited natural language processing
    \end{itemize}
    
    \item \textbf{Poor Mobile Experience}:
    \begin{itemize}
        \item Không responsive design
        \item Touch gesture support limited
        \item Performance issues on mobile devices
    \end{itemize}
    
    \item \textbf{Scalability Issues}:
    \begin{itemize}
        \item Monolithic architecture
        \item No cloud-native design
        \item Limited concurrent user support
    \end{itemize}
\end{enumerate}

\subsection{Pedagogical Gaps}
\label{subsec:pedagogical-gaps}

\begin{enumerate}
    \item \textbf{Lack of Learning Path}:
    \begin{itemize}
        \item No structured curriculum
        \item Random algorithm selection
        \item No prerequisite tracking
    \end{itemize}
    
    \item \textbf{Missing Assessment}:
    \begin{itemize}
        \item No knowledge evaluation
        \item Limited feedback mechanisms
        \item No progress tracking
    \end{itemize}
    
    \item \textbf{Community Absence}:
    \begin{itemize}
        \item No peer interaction
        \item Limited collaboration features
        \item No knowledge sharing platform
    \end{itemize}
\end{enumerate}

\section{Đóng góp của đồ án}
\label{sec:contribution}

Đồ án này đóng góp những điểm mới sau:

\begin{enumerate}
    \item \textbf{Comprehensive Platform}:
    \begin{itemize}
        \item Tích hợp visualizer, learning management, community
        \item End-to-end learning experience
        \item Modern technology stack
    \end{itemize}
    
    \item \textbf{AI-Powered Learning}:
    \begin{itemize}
        \item Multi-model AI integration (GPT + Gemini)
        \item Contextual help và code generation
        \item Personalized learning recommendations
    \end{itemize}
    
    \item \textbf{Community-Driven Approach}:
    \begin{itemize}
        \item Forum và Q\&A system
        \item Peer learning support
        \item Knowledge sharing platform
    \end{itemize}
    
    \item \textbf{Production-Ready Architecture}:
    \begin{itemize}
        \item Microservice design
        \item Cloud deployment
        \item Scalable và maintainable
    \end{itemize}
\end{enumerate}

\chapter{PHÂN TÍCH HỆ THỐNG}
\label{ch:system-analysis}

\section{Use Case Diagram}
\label{sec:use-case-diagram}

\subsection{Tổng quan Use Case}
\label{subsec:use-case-overview}

Hệ thống DSA Visualizer Platform phục vụ ba nhóm actor chính: Student, Instructor và Admin. Mỗi actor có các use case riêng biệt phù hợp với vai trò và quyền hạn của họ trong hệ thống.

\begin{center}
\textbf{[Use Case Diagram - System Overview]}\\
\textit{Diagram available in enhanced-diagrams/usecase-system-overview.drawio}
\end{center}

\subsection{Use Case chi tiết cho Learning Process}
\label{subsec:learning-process-usecase}

Quá trình học tập là core functionality của platform, bao gồm nhiều use case phức tạp với các interaction giữa Student và các subsystem khác nhau.

\begin{center}
\textbf{[Use Case Diagram - Detailed Scenarios]}\\
\textit{Diagram available in enhanced-diagrams/usecase-detailed-scenarios.drawio}
\end{center}

Các use case chính trong Learning Process:

\begin{enumerate}
    \item \textbf{Start Learning Session}: Khởi tạo session học tập mới
    \item \textbf{Select Algorithm}: Chọn thuật toán cần học
    \item \textbf{Study Theory}: Đọc tài liệu lý thuyết
    \item \textbf{Practice with Visualization}: Thực hành với animation
    \item \textbf{Solve Practice Problems}: Giải bài tập thực hành
    \item \textbf{Take Assessment}: Làm bài kiểm tra đánh giá
    \item \textbf{Get AI Assistance}: Nhận hỗ trợ từ AI assistant
    \item \textbf{Track Progress}: Theo dõi tiến độ học tập
\end{enumerate}

\subsection{Detailed Use Case Specifications}
\label{subsec:detailed-usecase-specs}

\subsubsection{UC001: Algorithm Visualization Learning}

\begin{longtable}{| p{3cm} | p{10cm} |}
\hline
\textbf{Use Case ID} & UC001 \\ \hline
\textbf{Tên Use Case} & Algorithm Visualization Learning \\ \hline
\textbf{Actor} & Student \\ \hline
\textbf{Mô tả ngắn gọn} & Học viên học thuật toán thông qua visualization interactive \\ \hline
\textbf{Trigger} & Học viên muốn học và hiểu thuật toán thông qua visualization \\ \hline
\textbf{Precondition} & 
\begin{itemize}
    \item Học viên đã đăng nhập vào hệ thống
    \item Hệ thống có sẵn algorithm content
    \item Browser hỗ trợ HTML5 Canvas/WebGL
\end{itemize} \\ \hline
\textbf{Luồng sự kiện chính} & 
\begin{enumerate}
    \item Học viên chọn loại thuật toán muốn học
    \item Hệ thống hiển thị danh sách algorithms available
    \item Học viên chọn specific algorithm (VD: Quick Sort)
    \item Hệ thống load algorithm visualizer interface
    \item Học viên input dữ liệu hoặc sử dụng sample data
    \item Học viên bắt đầu visualization process
    \item Hệ thống thực hiện step-by-step animation
    \item Học viên control speed, pause, resume theo nhu cầu
    \item Hệ thống hiển thị complexity analysis và explanation
    \item Học viên hoàn thành learning session
\end{enumerate} \\ \hline
\textbf{Luồng sự kiện thay thế} & 
\textbf{Alt 1:} Học viên muốn compare algorithms
\begin{itemize}
    \item Từ bước 3, học viên chọn multiple algorithms
    \item Hệ thống hiển thị comparison view
    \item Học viên chạy cùng lúc để so sánh performance
\end{itemize} \\ \hline
\textbf{Luồng ngoại lệ} & 
\textbf{Exc 1:} Input data không hợp lệ
\begin{itemize}
    \item Hệ thống hiển thị error message
    \item Yêu cầu học viên nhập lại data
\end{itemize}
\textbf{Exc 2:} Algorithm execution error
\begin{itemize}
    \item Hệ thống reset visualization
    \item Hiển thị default sample data
\end{itemize} \\ \hline
\textbf{Post Condition} & 
\begin{itemize}
    \item Learning progress được cập nhật
    \item Session data được lưu trong profile
    \item Analytics data được ghi nhận
\end{itemize} \\ \hline
\caption{Use Case Scenario: Algorithm Visualization Learning}
\label{tab:uc001} \\
\end{longtable}

\subsubsection{UC002: Interactive Algorithm Practice}

\begin{longtable}{| p{3cm} | p{10cm} |}
\hline
\textbf{Use Case ID} & UC002 \\ \hline
\textbf{Tên Use Case} & Interactive Algorithm Practice \\ \hline
\textbf{Actor} & Student \\ \hline
\textbf{Mô tả ngắn gọn} & Học viên thực hành thuật toán với interactive controls và custom input \\ \hline
\textbf{Trigger} & Học viên muốn thực hành để củng cố kiến thức thuật toán \\ \hline
\textbf{Precondition} & 
\begin{itemize}
    \item Học viên đã hoàn thành basic learning session
    \item Hệ thống có sẵn practice environment
    \item Practice mode được activate
\end{itemize} \\ \hline
\textbf{Luồng sự kiện chính} & 
\begin{enumerate}
    \item Học viên chọn Practice Mode từ main menu
    \item Hệ thống hiển thị available practice algorithms
    \item Học viên chọn algorithm để practice
    \item Hệ thống load interactive practice environment
    \item Học viên tạo custom input data hoặc chọn preset
    \item Học viên predict algorithm behavior trước khi execute
    \item Học viên execute algorithm step by step với controls
    \item Hệ thống provide real-time feedback và hints
    \item Học viên so sánh prediction với actual result
    \item Hệ thống tính performance score và suggestions
\end{enumerate} \\ \hline
\textbf{Luồng sự kiện thay thế} & 
\textbf{Alt 1:} Guided Practice Mode
\begin{itemize}
    \item Hệ thống provide hints và suggestions trong quá trình
    \item Học viên được hỗ trợ với detailed explanations
\end{itemize}
\textbf{Alt 2:} Challenge Mode
\begin{itemize}
    \item Hệ thống đưa ra specific challenges với time limits
    \item Học viên phải giải quyết trong thời gian giới hạn
\end{itemize} \\ \hline
\textbf{Luồng ngoại lệ} & 
\textbf{Exc 1:} Practice session timeout
\begin{itemize}
    \item Hệ thống auto-save current progress
    \item Cho phép học viên continue later từ checkpoint
\end{itemize}
\textbf{Exc 2:} Invalid practice input
\begin{itemize}
    \item Hệ thống validate input và show error
    \item Provide suggested valid input examples
\end{itemize} \\ \hline
\textbf{Post Condition} & 
\begin{itemize}
    \item Practice score được ghi nhận vào user profile
    \item Skill assessment metrics được cập nhật
    \item Achievement badges có thể được unlock
    \item Practice history được lưu cho future reference
\end{itemize} \\ \hline
\caption{Use Case Scenario: Interactive Algorithm Practice}
\label{tab:uc002} \\
\end{longtable}

\subsubsection{UC003: AI Assistant Consultation}

\begin{longtable}{| p{3cm} | p{10cm} |}
\hline
\textbf{Use Case ID} & UC003 \\ \hline
\textbf{Tên Use Case} & AI Assistant Consultation \\ \hline
\textbf{Actor} & Student \\ \hline
\textbf{Mô tả ngắn gọn} & Học viên sử dụng AI Assistant để được hỗ trợ học tập và giải đáp thắc mắc \\ \hline
\textbf{Trigger} & Học viên gặp khó khăn hoặc có câu hỏi cần giải đáp về algorithm \\ \hline
\textbf{Precondition} & 
\begin{itemize}
    \item Học viên đang trong learning session active
    \item AI Assistant service đang hoạt động và available
    \item Network connection stable cho real-time chat
\end{itemize} \\ \hline
\textbf{Luồng sự kiện chính} & 
\begin{enumerate}
    \item Học viên click vào AI Assistant icon trong interface
    \item Hệ thống mở AI chat interface với context loading
    \item Học viên nhập câu hỏi về algorithm hiện tại
    \item AI Assistant phân tích context và question intent
    \item AI generate comprehensive response với examples
    \item Hệ thống hiển thị answer với code examples và explanations
    \item Học viên có thể ask follow-up questions để clarify
    \item AI provide additional hints và learning resources nếu cần
    \item Học viên close AI Assistant khi satisfied với answers
\end{enumerate} \\ \hline
\textbf{Luồng sự kiện thay thế} & 
\textbf{Alt 1:} Code Analysis Request
\begin{itemize}
    \item Học viên paste existing code để AI review
    \item AI analyze code và suggest improvements với explanations
\end{itemize}
\textbf{Alt 2:} Algorithm Recommendation
\begin{itemize}
    \item Học viên mô tả specific problem cần giải quyết
    \item AI recommend suitable algorithms với comparison
\end{itemize} \\ \hline
\textbf{Luồng ngoại lệ} & 
\textbf{Exc 1:} AI service temporarily unavailable
\begin{itemize}
    \item Hệ thống hiển thị fallback resources và documentation
    \item Redirect đến static FAQ hoặc knowledge base
\end{itemize}
\textbf{Exc 2:} Question too complex hoặc ambiguous
\begin{itemize}
    \item AI request clarification với specific prompts
    \item Suggest breaking down question into smaller parts
\end{itemize} \\ \hline
\textbf{Post Condition} & 
\begin{itemize}
    \item Conversation history được lưu trong user session
    \item AI learning model được improve từ interaction
    \item User satisfaction feedback được collect tự động
    \item Related learning materials được suggest based on questions
\end{itemize} \\ \hline
\caption{Use Case Scenario: AI Assistant Consultation}
\label{tab:uc003} \\
\end{longtable}

\section{Class Diagram}
\label{sec:class-diagram}

\subsection{Tổng quan Class Diagram}
\label{subsec:class-overview}

Class diagram của hệ thống DSA Visualizer được thiết kế theo mô hình MVC (Model-View-Controller) và Clean Architecture, đảm bảo tính modular và scalability.

\begin{center}
\textbf{[Class Diagram - Core System]}\\
\textit{Diagram: class-diagram-clean.drawio}
\end{center}

\subsection{Các nhóm Class chính}

\subsubsection{User Management Classes}

\textbf{User Class:}
\begin{itemize}
    \item \textbf{Thuộc tính:} userID, email, username, password, role, createdAt, lastLogin
    \item \textbf{Phương thức:} login(), logout(), updateProfile(), changePassword()
    \item \textbf{Mối quan hệ:} User có nhiều LearningSession, có một UserProfile
\end{itemize}

\textbf{UserProfile Class:}
\begin{itemize}
    \item \textbf{Thuộc tính:} profileID, firstName, lastName, avatar, bio, preferences
    \item \textbf{Phương thức:} updatePersonalInfo(), setPreferences(), uploadAvatar()
    \item \textbf{Mối quan hệ:} Thuộc về một User, có nhiều Achievement
\end{itemize}

\subsubsection{Algorithm Visualization Classes}

\textbf{Algorithm Class:}
\begin{itemize}
    \item \textbf{Thuộc tính:} algorithmID, name, category, description, complexity, difficulty
    \item \textbf{Phương thức:} execute(), visualize(), getComplexity(), generateSteps()
    \item \textbf{Mối quan hệ:} Có nhiều AlgorithmStep, thuộc về một Category
\end{itemize}

\textbf{Visualizer Class:}
\begin{itemize}
    \item \textbf{Thuộc tính:} visualizerID, type, config, animationSpeed, currentStep
    \item \textbf{Phương thức:} start(), pause(), resume(), reset(), setSpeed()
    \item \textbf{Mối quan hệ:} Sử dụng Algorithm, tạo ra VisualizationSession
\end{itemize}

\textbf{AlgorithmStep Class:}
\begin{itemize}
    \item \textbf{Thuộc tính:} stepID, stepNumber, description, dataState, action
    \item \textbf{Phương thức:} execute(), undo(), getDescription(), visualize()
    \item \textbf{Mối quan hệ:} Thuộc về một Algorithm
\end{itemize}

\subsubsection{Learning Management Classes}

\textbf{LearningSession Class:}
\begin{itemize}
    \item \textbf{Thuộc tính:} sessionID, userID, algorithmID, startTime, endTime, score
    \item \textbf{Phương thức:} start(), complete(), calculateScore(), saveProgress()
    \item \textbf{Mối quan hệ:} Thuộc về User và Algorithm
\end{itemize}

\textbf{Progress Class:}
\begin{itemize}
    \item \textbf{Thuộc tính:} progressID, userID, totalSessions, completedAlgorithms, skillLevel
    \item \textbf{Phương thức:} updateProgress(), calculateSkillLevel(), getStatistics()
    \item \textbf{Mối quan hệ:} Thuộc về một User
\end{itemize}

\subsubsection{Assessment Classes}

\textbf{Quiz Class:}
\begin{itemize}
    \item \textbf{Thuộc tính:} quizID, title, description, questions, timeLimit, difficulty
    \item \textbf{Phương thức:} generateQuestions(), calculateScore(), validateAnswers()
    \item \textbf{Mối quan hệ:} Có nhiều Question, có nhiều QuizResult
\end{itemize}

\textbf{Question Class:}
\begin{itemize}
    \item \textbf{Thuộc tính:} questionID, content, options, correctAnswer, explanation
    \item \textbf{Phương thức:} validateAnswer(), getHint(), getExplanation()
    \item \textbf{Mối quan hệ:} Thuộc về một Quiz
\end{itemize}

\subsection{Design Patterns được sử dụng}

\subsubsection{Factory Pattern}
Sử dụng AlgorithmFactory để tạo ra các instance của different algorithm types:
\begin{itemize}
    \item SortingAlgorithmFactory
    \item SearchAlgorithmFactory  
    \item GraphAlgorithmFactory
\end{itemize}

\subsubsection{Observer Pattern}
VisualizationObserver được implement để notify UI components khi algorithm state changes:
\begin{itemize}
    \item ProgressObserver: Cập nhật progress bar
    \item AnimationObserver: Trigger animation effects
    \item ScoreObserver: Calculate và display scores
\end{itemize}

\subsubsection{Strategy Pattern}
Sử dụng cho algorithm execution strategies:
\begin{itemize}
    \item StepByStepStrategy: Execute từng bước
    \item ContinuousStrategy: Execute liên tục
    \item ComparisonStrategy: So sánh multiple algorithms
\end{itemize}

\section{Activity Diagram}
\label{sec:activity-diagram}

\subsection{Tổng quan Activity Diagram}
\label{subsec:activity-overview}

Activity diagram mô tả luồng hoạt động chính của hệ thống, từ khi user đăng nhập cho đến khi hoàn thành learning session.

\begin{center}
\textbf{[Activity Diagram - Learning Process]}\\
\textit{Diagram: activity-diagram-clean.drawio}
\end{center}

\subsection{Quy trình hoạt động chính}

\subsubsection{Authentication Flow}
\begin{enumerate}
    \item \textbf{Start:} User truy cập application
    \item \textbf{Decision:} Kiểm tra user đã login chưa?
    \item \textbf{False:} Redirect đến login page
    \item \textbf{Login Process:} User nhập credentials
    \item \textbf{Validation:} System validate user information
    \item \textbf{Decision:} Credentials có hợp lệ?
    \item \textbf{False:} Show error message, return to login
    \item \textbf{True:} Generate JWT token, redirect to dashboard
\end{enumerate}

\subsubsection{Algorithm Learning Flow}
\begin{enumerate}
    \item \textbf{Dashboard Access:} User vào main dashboard
    \item \textbf{Category Selection:} User chọn algorithm category
    \item \textbf{Algorithm Selection:} User chọn specific algorithm
    \item \textbf{Visualizer Loading:} System load algorithm visualizer
    \item \textbf{Input Configuration:} User configure input data
    \item \textbf{Decision:} User muốn start visualization?
    \item \textbf{True:} Begin algorithm execution
    \item \textbf{Step-by-step Execution:} System execute từng step
    \item \textbf{Animation Rendering:} Display visual animation
    \item \textbf{User Interaction:} User có thể pause/resume/adjust speed
    \item \textbf{Completion Check:} Algorithm execution complete?
    \item \textbf{False:} Continue next step
    \item \textbf{True:} Display final result và complexity analysis
\end{enumerate}

\subsubsection{AI Assistant Flow}
\begin{enumerate}
    \item \textbf{Trigger:} User click AI Assistant button
    \item \textbf{Context Collection:} System collect current learning context
    \item \textbf{Question Input:} User nhập question
    \item \textbf{NLP Processing:} AI analyze question intent
    \item \textbf{Knowledge Retrieval:} AI search relevant information
    \item \textbf{Response Generation:} AI generate appropriate response
    \item \textbf{Response Display:} System show AI response
    \item \textbf{Decision:} User có additional questions?
    \item \textbf{True:} Return to question input
    \item \textbf{False:} Close AI Assistant
\end{enumerate}

\subsubsection{Assessment Flow}
\begin{enumerate}
    \item \textbf{Quiz Selection:} User chọn quiz để làm
    \item \textbf{Quiz Loading:} System load quiz questions
    \item \textbf{Question Display:} Show current question
    \item \textbf{Answer Input:} User select/input answer
    \item \textbf{Answer Validation:} System validate answer
    \item \textbf{Feedback Display:} Show immediate feedback
    \item \textbf{Progress Update:} Update quiz progress
    \item \textbf{Decision:} Còn questions nào không?
    \item \textbf{True:} Next question
    \item \textbf{False:} Calculate final score
    \item \textbf{Result Display:} Show quiz results và recommendations
    \item \textbf{Progress Save:} Save user progress và achievements
\end{enumerate}

\subsection{Parallel Activities}

Hệ thống hỗ trợ các parallel activities:

\subsubsection{Background Services}
\begin{itemize}
    \item \textbf{Analytics Collection:} Continuous tracking user behavior
    \item \textbf{Performance Monitoring:} Real-time system performance tracking
    \item \textbf{Cache Management:} Background cache invalidation và refresh
    \item \textbf{Notification Processing:} Async notification sending
\end{itemize}

\subsubsection{Real-time Features}
\begin{itemize}
    \item \textbf{Live Progress Updates:} Real-time progress synchronization
    \item \textbf{Community Activity:} Live discussion forum updates
    \item \textbf{Collaborative Learning:} Multi-user learning sessions
\end{itemize}

\section{Sequence Diagram}
\label{sec:sequence-diagram}

\subsection{Tổng quan Sequence Diagram}
\label{subsec:sequence-overview}

Sequence diagram minh họa tương tác giữa các objects trong hệ thống theo thời gian, đặc biệt tập trung vào main learning scenarios.

\begin{center}
\textbf{[Sequence Diagram - Algorithm Learning Process]}\\
\textit{Diagram: sequence-diagram-clean.drawio}
\end{center}

\subsection{Chi tiết Sequence Interactions}

\subsubsection{Algorithm Visualization Sequence}

\textbf{Actors/Objects tham gia:}
\begin{itemize}
    \item Student (Actor)
    \item UI Controller
    \item Algorithm Service
    \item Visualizer Engine
    \item Database
    \item AI Assistant Service
\end{itemize}

\textbf{Sequence of Messages:}

\begin{enumerate}
    \item \textbf{Student → UI Controller:} selectAlgorithm(algorithmType)
    \item \textbf{UI Controller → Algorithm Service:} loadAlgorithm(algorithmType)
    \item \textbf{Algorithm Service → Database:} getAlgorithmDetails(algorithmType)
    \item \textbf{Database → Algorithm Service:} algorithmDetails
    \item \textbf{Algorithm Service → UI Controller:} algorithmLoaded
    \item \textbf{UI Controller → Student:} displayAlgorithmInterface()
    \item \textbf{Student → UI Controller:} configureInput(inputData)
    \item \textbf{UI Controller → Algorithm Service:} validateInput(inputData)
    \item \textbf{Algorithm Service → UI Controller:} inputValid
    \item \textbf{Student → UI Controller:} startVisualization()
    \item \textbf{UI Controller → Visualizer Engine:} initializeVisualization(algorithm, data)
    \item \textbf{Visualizer Engine → Algorithm Service:} executeStep()
    \item \textbf{Algorithm Service → Visualizer Engine:} stepResult
    \item \textbf{Visualizer Engine → UI Controller:} updateVisualization(stepResult)
    \item \textbf{UI Controller → Student:} displayAnimation()
    \item \textbf{Loop:} Repeat steps 12-15 until completion
    \item \textbf{Visualizer Engine → UI Controller:} visualizationComplete()
    \item \textbf{UI Controller → Database:} saveProgress(userId, sessionData)
    \item \textbf{UI Controller → Student:} displayResults(finalState, complexity)
\end{enumerate}

\subsubsection{AI Assistant Interaction Sequence}

\textbf{Sequence of Messages:}

\begin{enumerate}
    \item \textbf{Student → UI Controller:} openAIAssistant()
    \item \textbf{UI Controller → AI Assistant Service:} initializeSession(userId, context)
    \item \textbf{AI Assistant Service → Database:} getUserLearningContext(userId)
    \item \textbf{Database → AI Assistant Service:} learningContext
    \item \textbf{AI Assistant Service → UI Controller:} sessionReady
    \item \textbf{UI Controller → Student:} displayChatInterface()
    \item \textbf{Student → UI Controller:} askQuestion(question)
    \item \textbf{UI Controller → AI Assistant Service:} processQuestion(question, context)
    \item \textbf{AI Assistant Service:} analyzeIntent(question)
    \item \textbf{AI Assistant Service:} retrieveKnowledge(intent)
    \item \textbf{AI Assistant Service:} generateResponse(knowledge, context)
    \item \textbf{AI Assistant Service → UI Controller:} response
    \item \textbf{UI Controller → Student:} displayResponse(response)
    \item \textbf{AI Assistant Service → Database:} logInteraction(userId, question, response)
\end{enumerate}

\subsubsection{Assessment and Quiz Sequence}

\textbf{Sequence of Messages:}

\begin{enumerate}
    \item \textbf{Student → UI Controller:} selectQuiz(quizId)
    \item \textbf{UI Controller → Assessment Service:} loadQuiz(quizId)
    \item \textbf{Assessment Service → Database:} getQuizDetails(quizId)
    \item \textbf{Database → Assessment Service:} quizData
    \item \textbf{Assessment Service → UI Controller:} quizLoaded
    \item \textbf{UI Controller → Student:} displayQuizInterface()
    \item \textbf{Student → UI Controller:} startQuiz()
    \item \textbf{UI Controller → Assessment Service:} beginQuizSession(userId, quizId)
    \item \textbf{Loop for each question:}
    \begin{enumerate}
        \item \textbf{Assessment Service → UI Controller:} getNextQuestion()
        \item \textbf{UI Controller → Student:} displayQuestion(question)
        \item \textbf{Student → UI Controller:} submitAnswer(answer)
        \item \textbf{UI Controller → Assessment Service:} validateAnswer(questionId, answer)
        \item \textbf{Assessment Service → UI Controller:} answerResult(correct, explanation)
        \item \textbf{UI Controller → Student:} showFeedback(result, explanation)
    \end{enumerate}
    \item \textbf{Assessment Service → UI Controller:} calculateFinalScore()
    \item \textbf{UI Controller → Database:} saveQuizResult(userId, quizId, score, answers)
    \item \textbf{UI Controller → Student:} displayFinalResults(score, recommendations)
\end{enumerate}

\subsection{Error Handling Sequences}

\subsubsection{Authentication Error Sequence}
\begin{enumerate}
    \item \textbf{Student → UI Controller:} login(credentials)
    \item \textbf{UI Controller → Auth Service:} validateCredentials(credentials)
    \item \textbf{Auth Service → Database:} checkUserCredentials(credentials)
    \item \textbf{Database → Auth Service:} userNotFound/invalidPassword
    \item \textbf{Auth Service → UI Controller:} authenticationFailed(errorType)
    \item \textbf{UI Controller → Student:} displayErrorMessage(errorType)
    \item \textbf{UI Controller → Student:} requestCredentialsAgain()
\end{enumerate}

\subsubsection{System Error Recovery Sequence}
\begin{enumerate}
    \item \textbf{Any Service:} systemError(errorDetails)
    \item \textbf{Error Handler:} logError(errorDetails)
    \item \textbf{Error Handler → Monitoring Service:} reportError(errorDetails)
    \item \textbf{Error Handler → UI Controller:} notifyUser(genericErrorMessage)
    \item \textbf{UI Controller → Student:} displayErrorScreen(recoveryOptions)
    \item \textbf{Error Handler:} attemptRecovery()
    \item \textbf{Fallback Service:} provideFallbackFunctionality()
\end{enumerate}

\section{System Architecture Analysis}
\label{sec:system-architecture}

\subsection{Tổng quan Architecture}
\label{subsec:architecture-overview}

Hệ thống DSA Visualizer được thiết kế theo mô hình 5-layer architecture để đảm bảo scalability, maintainability và performance optimization.

\begin{center}
\textbf{[System Architecture Diagram]}\\
\textit{Diagram: system-architecture.drawio}
\end{center}

\subsection{Chi tiết các Layer}

\subsubsection{UI Layer (Presentation Layer)}
\textbf{Công nghệ sử dụng:} Next.js 14, React 18, TypeScript, TailwindCSS

\textbf{Thành phần chính:}
\begin{itemize}
    \item \textbf{Interactive Visualizers:} Canvas-based algorithm animations
    \item \textbf{Control Panels:} Speed control, step-by-step navigation
    \item \textbf{Dashboard Interface:} User progress tracking và statistics
    \item \textbf{AI Chat Interface:} Real-time chat với AI Assistant
    \item \textbf{Assessment Interface:} Quiz và practice exercises
\end{itemize}

\textbf{Design Patterns:}
\begin{itemize}
    \item Component-based architecture
    \item State management với Context API
    \item Custom hooks cho reusable logic
    \item Responsive design patterns
\end{itemize}

\subsubsection{Visualization Engine Layer}
\textbf{Công nghệ sử dụng:} D3.js, Canvas API, WebGL

\textbf{Core Components:}
\begin{itemize}
    \item \textbf{Animation Controller:} Quản lý animation timeline và state
    \item \textbf{Rendering Engine:} High-performance visualization rendering
    \item \textbf{Interaction Handler:} User input processing cho visualizations
    \item \textbf{State Manager:} Algorithm state tracking và history
\end{itemize}

\textbf{Visualization Types:}
\begin{itemize}
    \item Array/List visualizations với color coding
    \item Tree structures với interactive nodes
    \item Graph visualizations với edge animations
    \item Comparison views cho multiple algorithms
\end{itemize}

\subsubsection{Backend Services Layer}
\textbf{Công nghệ sử dụng:} Node.js, Express.js, TypeScript

\textbf{Microservices Architecture:}
\begin{itemize}
    \item \textbf{Algorithm Service:} Algorithm execution và step generation
    \item \textbf{User Service:} Authentication, profile management
    \item \textbf{Learning Service:} Progress tracking, session management
    \item \textbf{Assessment Service:} Quiz generation, scoring system
    \item \textbf{AI Service:} Natural language processing, knowledge retrieval
    \item \textbf{Analytics Service:} User behavior tracking, performance metrics
\end{itemize}

\textbf{API Design:}
\begin{itemize}
    \item RESTful APIs với OpenAPI documentation
    \item GraphQL endpoints cho complex data queries
    \item WebSocket connections cho real-time features
    \item Rate limiting và security middleware
\end{itemize}

\subsubsection{Data Management Layer}
\textbf{Công nghệ sử dụng:} MongoDB, Redis, PostgreSQL

\textbf{Database Strategy:}
\begin{itemize}
    \item \textbf{MongoDB:} User profiles, learning sessions, algorithm metadata
    \item \textbf{PostgreSQL:} Structured data, analytics, reporting
    \item \textbf{Redis:} Session caching, real-time data, leaderboards
\end{itemize}

\textbf{Data Models:}
\begin{itemize}
    \item User và Profile entities với relationship mapping
    \item Algorithm metadata với complexity analysis
    \item Learning progress với detailed tracking
    \item Assessment results với statistical analysis
\end{itemize}

\subsubsection{Infrastructure Layer}
\textbf{Deployment Strategy:} Docker containers, Kubernetes orchestration

\textbf{Cloud Services:}
\begin{itemize}
    \item \textbf{Compute:} Auto-scaling web servers
    \item \textbf{Storage:} Distributed file storage cho assets
    \item \textbf{CDN:} Global content delivery network
    \item \textbf{Monitoring:} Application performance monitoring
\end{itemize}

\textbf{Security Measures:}
\begin{itemize}
    \item JWT-based authentication với refresh tokens
    \item HTTPS enforcement với SSL certificates
    \item Input validation và SQL injection prevention
    \item Cross-Origin Resource Sharing (CORS) configuration
\end{itemize}

\subsection{Integration Patterns}

\subsubsection{Event-Driven Architecture}
\begin{itemize}
    \item User action events trigger visualization updates
    \item Progress events update learning analytics
    \item Achievement events trigger notification system
    \item Error events activate monitoring và alerting
\end{itemize}

\subsubsection{Caching Strategy}
\begin{itemize}
    \item Browser caching cho static assets
    \item Redis caching cho frequently accessed data
    \item CDN caching cho global performance
    \item Application-level caching cho computed results
\end{itemize}

\section{Design Principles và Best Practices}
\label{sec:design-principles}

\subsection{SOLID Principles Implementation}

\subsubsection{Single Responsibility Principle}
Mỗi class và component có một responsibility duy nhất:
\begin{itemize}
    \item VisualizationRenderer chỉ handle rendering logic
    \item AlgorithmExecutor chỉ handle algorithm execution
    \item UserManager chỉ handle user-related operations
\end{itemize}

\subsubsection{Open/Closed Principle}
Hệ thống được thiết kế để extend functionality without modification:
\begin{itemize}
    \item Plugin architecture cho new algorithm types
    \item Extension system cho custom visualizations
    \item Configurable assessment frameworks
\end{itemize}

\subsubsection{Liskov Substitution Principle}
Abstract classes và interfaces đảm bảo substitutability:
\begin{itemize}
    \item Algorithm interface có thể được implement bởi any algorithm type
    \item Visualizer interface support multiple rendering strategies
    \item Assessment interface accommodate different quiz types
\end{itemize}

\subsection{Performance Optimization}

\subsubsection{Frontend Optimization}
\begin{itemize}
    \item Code splitting và lazy loading cho components
    \item Memoization cho expensive computations
    \item Virtual scrolling cho large datasets
    \item Debouncing cho user input handling
\end{itemize}

\subsubsection{Backend Optimization}
\begin{itemize}
    \item Database query optimization với proper indexing
    \item Connection pooling cho database connections
    \item Asynchronous processing cho time-consuming tasks
    \item Load balancing cho high availability
\end{itemize}

\subsection{Accessibility và Usability}

\subsubsection{Accessibility Features}
\begin{itemize}
    \item WCAG 2.1 compliance cho accessibility standards
    \item Keyboard navigation support
    \item Screen reader compatibility
    \item High contrast mode cho visual impairments
\end{itemize}

\subsubsection{Usability Features}
\begin{itemize}
    \item Intuitive user interface design
    \item Progressive disclosure của complex features
    \item Contextual help và tooltips
    \item Responsive design cho multiple devices
\end{itemize}

\section{Kết luận Chapter 3}
\label{sec:chapter3-conclusion}

Chapter 3 đã phân tích chi tiết hệ thống DSA Visualizer từ góc độ technical architecture và design. Các điểm chính bao gồm:

\subsection{Use Case Analysis}
Đã định nghĩa và mô tả chi tiết các use cases chính của hệ thống, bao gồm algorithm learning, interactive practice, và AI assistant consultation. Mỗi use case được documented với format table chi tiết theo chuẩn academic.

\subsection{UML Diagrams Analysis}
\begin{itemize}
    \item \textbf{Class Diagram:} Thiết kế OOP với các design patterns phù hợp
    \item \textbf{Activity Diagram:} Mô tả chi tiết quy trình hoạt động và decision flows
    \item \textbf{Sequence Diagram:} Phân tích tương tác giữa objects theo timeline
\end{itemize}

\subsection{System Architecture}
Thiết kế 5-layer architecture đảm bảo:
\begin{itemize}
    \item Scalability cho future expansion
    \item Maintainability với modular design
    \item Performance optimization với caching strategies
    \item Security với comprehensive protection measures
\end{itemize}

\subsection{Technical Excellence}
Áp dụng SOLID principles, design patterns, và best practices để tạo ra một hệ thống robust và professional.

Tiếp theo, Chapter 4 sẽ focus vào implementation details và technical specifications của từng component.

\subsubsection{Use Case: Get AI Assistance}

\begin{longtable}{| p{3cm} | p{10cm} |}
\hline
\textbf{Use Case ID} & UC-AI-001 \\ \hline
\textbf{Tên Use Case} & Get AI Assistance \\ \hline
\textbf{Actor} & Student \\ \hline
\textbf{Mô tả ngắn gọn} & Student nhận hỗ trợ từ AI assistant để hiểu thuật toán hoặc giải quyết vấn đề \\ \hline
\textbf{Trigger} & Student gặp khó khăn và cần hỗ trợ trong quá trình học \\ \hline
\textbf{Precondition} & 
\begin{itemize}
    \item Student đã đăng nhập vào hệ thống
    \item AI service available và responsive
    \item Current learning context được load
\end{itemize} \\ \hline
\textbf{Luồng sự kiện chính} & 
\begin{enumerate}
    \item Student click "AI Assistant" button trong interface
    \item System mở AI chat interface với context loading
    \item Student nhập câu hỏi hoặc chọn suggested questions
    \item System gửi request đến AI service kèm theo context
    \item AI service process request và trả về comprehensive response
    \item System hiển thị AI response với proper formatting
    \item Student có thể tiếp tục conversation với follow-up questions
    \item System tự động lưu conversation history
\end{enumerate} \\ \hline
\textbf{Luồng sự kiện thay thế} & 
\textbf{Alt 1:} Code Generation Request
\begin{itemize}
    \item Student request code implementation cho current algorithm
    \item Student chọn programming language preference
    \item AI generate syntax-highlighted code với detailed explanation
\end{itemize}
\textbf{Alt 2:} Step-by-step Explanation
\begin{itemize}
    \item Student click "Explain Current Step" trong visualization
    \item AI explain current step synchronized với animation
\end{itemize} \\ \hline
\textbf{Luồng ngoại lệ} & 
\textbf{Exc 1:} AI Service Unavailable
\begin{itemize}
    \item AI service timeout hoặc connection error
    \item System show fallback static help resources
    \item Log error cho admin notification
\end{itemize}
\textbf{Exc 2:} Rate Limit Exceeded
\begin{itemize}
    \item Too many requests từ user trong short period
    \item System show rate limit message với countdown
    \item Suggest alternative help resources
\end{itemize} \\ \hline
\textbf{Post Condition} & 
\begin{itemize}
    \item Conversation history được lưu trong user profile
    \item AI usage statistics được cập nhật cho analytics
    \item Learning context được enrich từ conversation
    \item User satisfaction metrics được collect
\end{itemize} \\ \hline
\caption{Use Case Scenario: Get AI Assistance}
\label{tab:uc-ai-assistance} \\
\end{longtable}

\section{Class Diagram}
\label{sec:class-diagram}

\subsection{Core Domain Classes}
\label{subsec:core-classes}

Hệ thống được thiết kế theo Domain-Driven Design với các core domain classes sau:

\begin{figure}[H]
\centering
\includegraphics[width=1.0\textwidth]{enhanced-diagrams/class-diagram-clean.drawio}
\caption{Class Diagram cho Core Domain}
\label{fig:class-core}
\end{figure}

\subsubsection{User Management Classes}

\begin{itemize}
    \item \textbf{User}: Base class cho tất cả users
    \item \textbf{Student}: Extends User, thêm learning-specific attributes
    \item \textbf{Instructor}: Extends User, thêm teaching-specific attributes  
    \item \textbf{Admin}: Extends User, thêm system management capabilities
\end{itemize}

\subsubsection{Learning Domain Classes}

\begin{itemize}
    \item \textbf{Algorithm}: Represents thuật toán với metadata
    \item \textbf{Visualization}: Chứa animation data và configuration
    \item \textbf{LearningSession}: Tracks user interaction với algorithm
    \item \textbf{Progress}: Theo dõi learning progress của user
    \item \textbf{Assessment}: Quiz và evaluation system
\end{itemize}

\subsection{Service Layer Classes}
\label{subsec:service-classes}

\begin{figure}[H]
\centering
\includegraphics[width=1.0\textwidth]{enhanced-diagrams/class-diagram-clean.drawio}
\caption{Class Diagram cho Service Layer}
\label{fig:class-services}
\end{figure}

\subsubsection{Visualization Services}

\begin{itemize}
    \item \textbf{VisualizationEngine}: Core engine cho animation rendering
    \item \textbf{AnimationController}: Điều khiển animation playback
    \item \textbf{DataProcessor}: Xử lý input data cho visualization
    \item \textbf{RenderingService}: Abstract layer cho different renderers
\end{itemize}

\subsubsection{AI Services}

\begin{itemize}
    \item \textbf{AIAssistant}: Main interface cho AI interactions
    \item \textbf{OpenAIService}: Integration với OpenAI GPT
    \item \textbf{GeminiService}: Integration với Google Gemini
    \item \textbf{ContextBuilder}: Xây dựng context cho AI requests
\end{itemize}

\section{Activity Diagram}
\label{sec:activity-diagram}

\subsection{Learning Process Flow}
\label{subsec:learning-flow}

Activity diagram sau mô tả complete learning process từ khi student bắt đầu học một thuật toán mới:

\begin{figure}[H]
\centering
\includegraphics[width=0.8\textwidth]{enhanced-diagrams/activity-diagram-clean.drawio}
\caption{Activity Diagram cho Learning Process}
\label{fig:activity-learning}
\end{figure}

\subsubsection{Main Activities}

\begin{enumerate}
    \item \textbf{Algorithm Selection}: Student chọn thuật toán từ library
    \item \textbf{Theory Study}: Đọc theoretical background
    \item \textbf{Visualization Practice}: Tương tác với animated visualization
    \item \textbf{Code Understanding}: Phân tích implementation code
    \item \textbf{Problem Solving}: Áp dụng thuật toán vào bài tập
    \item \textbf{Assessment}: Đánh giá mức độ hiểu biết
    \item \textbf{Progress Update}: Cập nhật learning progress
\end{enumerate}

\subsubsection{Decision Points}

\begin{itemize}
    \item \textbf{Prerequisite Check}: Kiểm tra điều kiện tiên quyết
    \item \textbf{Difficulty Assessment}: Đánh giá độ khó phù hợp
    \item \textbf{Understanding Verification}: Xác nhận mức độ hiểu
    \item \textbf{Need Help Decision}: Quyết định có cần AI assistance
\end{itemize}

\subsection{AI Assistance Flow}
\label{subsec:ai-flow}

\begin{figure}[H]
\centering
\includegraphics[width=0.7\textwidth]{enhanced-diagrams/activity-diagram-clean.drawio}
\caption{Activity Diagram cho AI Assistance Flow}
\label{fig:activity-ai}
\end{figure}

Process khi student request AI help:

\begin{enumerate}
    \item \textbf{Context Collection}: Gather current learning context
    \item \textbf{Query Processing}: Process natural language query
    \item \textbf{AI Model Selection}: Chọn appropriate AI model
    \item \textbf{Response Generation}: Generate contextual response
    \item \textbf{Response Formatting}: Format cho display
    \item \textbf{Feedback Collection}: Thu thập user feedback
\end{enumerate}

\section{Sequence Diagram}
\label{sec:sequence-diagram}

\subsection{Visualization Rendering Sequence}
\label{subsec:visualization-sequence}

Sequence diagram sau mô tả interaction giữa các components khi render visualization:

\begin{figure}[H]
\centering
\includegraphics[width=1.0\textwidth]{enhanced-diagrams/sequence-diagram-clean.drawio}
\caption{Sequence Diagram cho Visualization Rendering}
\label{fig:sequence-viz}
\end{figure}

\subsubsection{Participants}

\begin{itemize}
    \item \textbf{Student}: User initiating visualization
    \item \textbf{VisualizationUI}: Frontend visualization component
    \item \textbf{VisualizationEngine}: Core rendering engine
    \item \textbf{AnimationController}: Animation management
    \item \textbf{DataProcessor}: Input data processing
    \item \textbf{ProgressTracker}: Learning progress tracking
\end{itemize}

\subsubsection{Key Interactions}

\begin{enumerate}
    \item Student selects algorithm và input parameters
    \item UI validates input và sends request to engine
    \item Engine processes data và prepares animation steps
    \item AnimationController manages playback timing
    \item Progress tracking updates throughout process
    \item Final state saved to user progress
\end{enumerate}

\subsection{AI Assistant Interaction Sequence}
\label{subsec:ai-sequence}

\begin{figure}[H]
\centering
\includegraphics[width=1.0\textwidth]{enhanced-diagrams/sequence-diagram-clean.drawio}
\caption{Sequence Diagram cho AI Assistant Interaction}
\label{fig:sequence-ai}
\end{figure}

\subsubsection{Multi-Model AI Architecture}

Platform sử dụng multiple AI models để optimize cho different use cases:

\begin{enumerate}
    \item \textbf{OpenAI GPT}: Cho natural language explanations
    \item \textbf{Google Gemini}: Cho code generation và analysis
    \item \textbf{Context Router}: Intelligent routing dựa trên query type
    \item \textbf{Response Aggregator}: Combine responses từ multiple models
\end{enumerate}

\subsection{Community Interaction Sequence}
\label{subsec:community-sequence}

\begin{figure}[H]
\centering
\includegraphics[width=1.0\textwidth]{enhanced-diagrams/sequence-diagram-clean.drawio}
\caption{Sequence Diagram cho Community Features}
\label{fig:sequence-community}
\end{figure}

Mô tả interaction trong forum và Q\&A system:

\begin{enumerate}
    \item User posts question hoặc discussion topic
    \item System validates content và applies moderation rules
    \item Notification service alerts relevant users
    \item Other users provide answers và comments
    \item Voting system ranks responses
    \item AI assistant có thể provide supplementary answers
    \item Final resolution updates knowledge base
\end{enumerate}

\include{chapters/4-system-design}
\include{chapters/5-implementation}
\include{chapters/6-testing-evaluation}
\include{chapters/7-conclusion}

\nocite{*}
\addcontentsline{toc}{chapter}{References}
\printbibliography
\include{chapters/appendix}
\end{document}
\end{enumerate}

\subsection{Đối tượng sử dụng}

\begin{table}[H]
\centering
\caption{Phân loại người dùng hệ thống}
\begin{tabular}{|l|p{10cm}|}
\hline
\textbf{Nhóm người dùng} & \textbf{Mô tả và nhu cầu} \\
\hline
\textbf{Sinh viên} & Học tập các thuật toán cơ bản, thực hành coding, tham gia cộng đồng \\
\hline
\textbf{Giảng viên} & Sử dụng làm công cụ giảng dạy, tạo nội dung học tập, quản lý lớp học \\
\hline
\textbf{Lập trình viên} & Ôn tập thuật toán cho phỏng vấn, nghiên cứu thuật toán mới \\
\hline
\textbf{Người tự học} & Tìm hiểu thuật toán một cách tự động, có AI hỗ trợ \\
\hline
\textbf{Admin} & Quản lý hệ thống, theo dõi usage, xử lý feedback \\
\hline
\end{tabular}
\end{table}

\section{Phân tích yêu cầu}

\subsection{Yêu cầu chức năng}

\subsubsection{Hệ thống xác thực và phân quyền}
\begin{itemize}
    \item \textbf{RF-01}: Đăng ký/đăng nhập với email, Google, GitHub
    \item \textbf{RF-02}: Hệ thống phân quyền 4 cấp (Guest, User, Teacher, Admin)
    \item \textbf{RF-03}: Quản lý profile và theo dõi tiến độ học tập
    \item \textbf{RF-04}: Password reset và email verification
\end{itemize}

\subsubsection{Visualizer thuật toán}
\begin{itemize}
    \item \textbf{RF-05}: Trực quan hóa sorting algorithms (Quick, Merge, Heap, Radix, etc.)
    \item \textbf{RF-06}: Visualizer cho data structures (Binary Tree, AVL, Hash Table, etc.)
    \item \textbf{RF-07}: Graph algorithms (Dijkstra, BFS, DFS, MST)
    \item \textbf{RF-08}: Dynamic Programming và Pathfinding visualizations
    \item \textbf{RF-09}: Điều khiển tốc độ animation và step-by-step execution
    \item \textbf{RF-10}: Multiple view modes và theme options
\end{itemize}

\subsubsection{AI Learning Assistant}
\begin{itemize}
    \item \textbf{RF-11}: Code analysis và suggestions cho 6 ngôn ngữ
    \item \textbf{RF-12}: Interactive explanations và step-by-step guidance
    \item \textbf{RF-13}: Algorithm optimization recommendations
    \item \textbf{RF-14}: Real-time code feedback và error detection
\end{itemize}

\subsubsection{Community Features}
\begin{itemize}
    \item \textbf{RF-15}: Discussion forums với threading và moderation
    \item \textbf{RF-16}: Q\&A system kiểu Stack Overflow
    \item \textbf{RF-17}: Rating và voting system
    \item \textbf{RF-18}: Real-time notifications và comments
\end{itemize}

\subsubsection{Admin Dashboard}
\begin{itemize}
    \item \textbf{RF-19}: User management và role assignment
    \item \textbf{RF-20}: Analytics và usage tracking
    \item \textbf{RF-21}: Feedback management và response system
    \item \textbf{RF-22}: System monitoring và performance metrics
\end{itemize}

\subsection{Yêu cầu phi chức năng}

\subsubsection{Hiệu năng}
\begin{itemize}
    \item \textbf{NFR-01}: Page load time < 2 giây
    \item \textbf{NFR-02}: Animation frame rate >= 60 FPS
    \item \textbf{NFR-03}: API response time < 500ms
    \item \textbf{NFR-04}: Support đồng thời 1000+ concurrent users
\end{itemize}

\subsubsection{Khả năng sử dụng}
\begin{itemize}
    \item \textbf{NFR-05}: Responsive design cho mobile, tablet, desktop
    \item \textbf{NFR-06}: Accessibility theo chuẩn WCAG 2.1 AA
    \item \textbf{NFR-07}: Multi-language support (Vietnamese, English)
    \item \textbf{NFR-08}: Dark/Light mode với system preference detection
\end{itemize}

\subsubsection{Bảo mật}
\begin{itemize}
    \item \textbf{NFR-09}: HTTPS encryption cho tất cả communications
    \item \textbf{NFR-10}: JWT token-based authentication
    \item \textbf{NFR-11}: Input sanitization và SQL injection prevention
    \item \textbf{NFR-12}: Rate limiting cho API endpoints
\end{itemize}

\subsubsection{Khả năng mở rộng}
\begin{itemize}
    \item \textbf{NFR-13}: Modular architecture cho phép add algorithms mới
    \item \textbf{NFR-14}: Database scalability với connection pooling
    \item \textbf{NFR-15}: Serverless deployment trên Vercel
    \item \textbf{NFR-16}: CDN integration cho static assets
\end{itemize}

\section{Thiết kế hệ thống}

\subsection{Kiến trúc tổng quan}

DSA Visualizer Platform được xây dựng theo kiến trúc phân tầng modular với các thành phần được tách biệt rõ ràng:

\begin{figure}[H]
\centering
\includegraphics[width=0.9\textwidth]{diagrams/system_architecture.pdf}
\caption{Kiến trúc tổng thể hệ thống DSA Visualizer}
\label{fig:system_architecture}
\end{figure}

\subsubsection{Presentation Layer (Frontend)}
\begin{itemize}
    \item \textbf{Next.js 15 App Router}: Framework chính với server-side rendering
    \item \textbf{React 19}: UI library với concurrent features và hooks mới
    \item \textbf{TypeScript}: Type safety và developer experience
    \item \textbf{Tailwind CSS + shadcn/ui}: Modern styling với component library
    \item \textbf{Framer Motion}: Animation library cho smooth transitions
\end{itemize}

\subsubsection{Business Logic Layer}
\begin{itemize}
    \item \textbf{Algorithm Engines}: Core logic cho 24+ visualization algorithms
    \item \textbf{AI Integration}: OpenAI GPT và Google Gemini APIs
    \item \textbf{Authentication}: NextAuth.js với multi-provider support
    \item \textbf{State Management}: React Context và custom hooks
\end{itemize}

\subsubsection{Data Access Layer}
\begin{itemize}
    \item \textbf{Prisma ORM}: Type-safe database operations
    \item \textbf{API Routes}: Next.js serverless functions
    \item \textbf{Database}: PostgreSQL với connection pooling
    \item \textbf{Caching}: Redis cho session và API response caching
\end{itemize}

\subsection{Use Case Diagram}

\begin{figure}[H]
\centering
\includegraphics[width=0.9\textwidth]{diagrams/usecase_main.pdf}
\caption{Use Case Diagram - Các chức năng chính của hệ thống}
\label{fig:usecase}
\end{figure}

Use case diagram mô tả các chức năng chính của hệ thống cho từng nhóm người dùng:

\subsubsection{Guest User}
\begin{itemize}
    \item Xem basic algorithm visualizations
    \item Đăng ký tài khoản mới
    \item Truy cập public learning materials
\end{itemize}

\subsubsection{Authenticated User}
\begin{itemize}
    \item Full access đến tất cả visualizers
    \item Sử dụng AI Learning Assistant
    \item Tham gia community forums và Q\&A
    \item Track learning progress
    \item Customize preferences và themes
\end{itemize}

\subsubsection{Teacher}
\begin{itemize}
    \item Tạo và quản lý learning content
    \item Access advanced analytics
    \item Moderate discussions
    \item Create custom algorithm examples
\end{itemize}

\subsubsection{Admin}
\begin{itemize}
    \item User management và role assignment
    \item System monitoring và performance analytics
    \item Content moderation và feedback management
    \item System configuration và maintenance
\end{itemize}

\subsection{Class Diagram}

\begin{figure}[H]
\centering
\includegraphics[width=0.9\textwidth]{diagrams/class_diagram.pdf}
\caption{Class Diagram - Cấu trúc lớp chính của hệ thống}
\label{fig:class_diagram}
\end{figure}

Class diagram thể hiện các entities chính và relationships:

\subsubsection{User Management}
\begin{itemize}
    \item \textbf{User}: Base user entity với authentication info
    \item \textbf{Profile}: Extended user information và preferences
    \item \textbf{Role}: Role-based access control system
    \item \textbf{Session}: User session management
\end{itemize}

\subsubsection{Learning System}
\begin{itemize}
    \item \textbf{Algorithm}: Metadata về các thuật toán
    \item \textbf{Visualization}: Visualization instances và state
    \item \textbf{Progress}: User learning progress tracking
    \item \textbf{Tutorial}: Step-by-step learning content
\end{itemize}

\subsubsection{Community Features}
\begin{itemize}
    \item \textbf{Discussion}: Forum discussions và threads
    \item \textbf{Question}: Q\&A system entries
    \item \textbf{Answer}: Responses đến questions
    \item \textbf{Vote}: Voting system cho quality control
\end{itemize}

\section{Phân tích thiết kế chi tiết}

\subsection{Activity Diagram}

\begin{figure}[H]
\centering
\includegraphics[width=0.9\textwidth]{diagrams/activity_diagram.pdf}
\caption{Activity Diagram - Quy trình học tập với AI assistance}
\label{fig:activity_diagram}
\end{figure}

Activity diagram mô tả quy trình học tập chính của user:

\subsubsection{Giai đoạn khởi tạo}
\begin{enumerate}
    \item User truy cập platform
    \item Hệ thống check authentication status
    \item Load user preferences và progress
    \item Initialize visualization environment
\end{enumerate}

\subsubsection{Giai đoạn chọn thuật toán}
\begin{enumerate}
    \item Browse algorithm categories
    \item Select specific algorithm
    \item Load visualization component
    \item Display algorithm information
\end{enumerate}

\subsubsection{Giai đoạn visualization}
\begin{enumerate}
    \item Configure input parameters
    \item Start algorithm execution
    \item Display step-by-step animation
    \item Provide real-time explanations
    \item Track user interactions
\end{enumerate}

\subsubsection{Giai đoạn AI assistance}
\begin{enumerate}
    \item User requests help hoặc explanation
    \item AI analyzes current context
    \item Generate personalized response
    \item Display interactive code examples
    \item Update learning progress
\end{enumerate}

\subsection{Sequence Diagram}

\begin{figure}[H]
\centering
\includegraphics[width=0.9\textwidth]{diagrams/sequence_diagram.pdf}
\caption{Sequence Diagram - Tương tác giữa các components chính}
\label{fig:sequence_diagram}
\end{figure}

Sequence diagram mô tả tương tác giữa các thành phần trong quá trình visualization:

\subsubsection{Khởi tạo visualization}
\begin{enumerate}
    \item \textbf{User Interface} → \textbf{Algorithm Controller}: Request algorithm execution
    \item \textbf{Algorithm Controller} → \textbf{Algorithm Engine}: Initialize algorithm với input data
    \item \textbf{Algorithm Engine} → \textbf{Step Generator}: Generate step-by-step execution plan
    \item \textbf{Step Generator} → \textbf{Animation Controller}: Prepare animation sequences
\end{enumerate}

\subsubsection{Execution và rendering}
\begin{enumerate}
    \item \textbf{Animation Controller} → \textbf{Renderer}: Start animation loop
    \item \textbf{Renderer} → \textbf{UI Components}: Update visual elements
    \item \textbf{UI Components} → \textbf{User Interface}: Display current state
    \item \textbf{Algorithm Engine} → \textbf{Progress Tracker}: Update completion status
\end{enumerate}

\subsubsection{AI interaction}
\begin{enumerate}
    \item \textbf{User Interface} → \textbf{AI Assistant}: Request explanation
    \item \textbf{AI Assistant} → \textbf{AI API}: Send context và query
    \item \textbf{AI API} → \textbf{AI Assistant}: Return generated response
    \item \textbf{AI Assistant} → \textbf{User Interface}: Display explanation
\end{enumerate}

\section{Cài đặt và triển khai}

\subsection{Công nghệ sử dụng}

\begin{table}[H]
\centering
\caption{Stack công nghệ chi tiết}
\begin{tabular}{|l|l|l|}
\hline
\textbf{Category} & \textbf{Technology} & \textbf{Version} \\
\hline
\multirow{6}{*}{Frontend} & Next.js & 15.0.0 \\
\cline{2-3}
 & React & 19.0.0 \\
\cline{2-3}
 & TypeScript & 5.6.0 \\
\cline{2-3}
 & Tailwind CSS & 3.4.0 \\
\cline{2-3}
 & Framer Motion & 11.11.0 \\
\cline{2-3}
 & shadcn/ui & Latest \\
\hline
\multirow{4}{*}{Backend} & Next.js API Routes & 15.0.0 \\
\cline{2-3}
 & Prisma ORM & 5.21.0 \\
\cline{2-3}
 & NextAuth.js & 4.24.0 \\
\cline{2-3}
 & PostgreSQL & 15+ \\
\hline
\multirow{3}{*}{AI Integration} & OpenAI API & GPT-4 \\
\cline{2-3}
 & Google Gemini & Pro \\
\cline{2-3}
 & Custom Prompting & - \\
\hline
\multirow{3}{*}{Development} & ESLint & 9.0.0 \\
\cline{2-3}
 & Prettier & Latest \\
\cline{2-3}
 & Husky & Git hooks \\
\hline
\multirow{2}{*}{Deployment} & Vercel & Platform \\
\cline{2-3}
 & GitHub Actions & CI/CD \\
\hline
\end{tabular}
\end{table}

\subsection{Cấu trúc dự án}

\begin{lstlisting}[language=bash, caption=Cấu trúc thư mục dự án]
dsa-visualizer/
+-- src/
|   +-- app/                    # Next.js 15 App Router
|   |   +-- globals.css         # Global styles
|   |   +-- layout.tsx          # Root layout
|   |   +-- page.tsx           # Homepage
|   |   +-- dashboard/         # User dashboard
|   |   +-- sorting/           # Sorting algorithms page
|   |   +-- pathfinding/       # Pathfinding visualization
|   |   +-- visualizer/        # Individual algorithm pages
|   |       +-- binary-tree/   # Binary tree visualizer
|   |       +-- graph/         # Graph algorithms
|   |       +-- sorting/       # Sorting visualizers
|   |       +-- [algorithm]/   # Dynamic algorithm routes
|   +-- components/            # React components
|   |   +-- ui/               # Base UI components (shadcn/ui)
|   |   +-- visualizers/      # Algorithm visualization components
|   |   +-- navigation/       # Navigation components
|   |   +-- landing/          # Landing page components
|   |   +-- shared/           # Shared utility components
|   +-- hooks/                # Custom React hooks
|   |   +-- use-mobile.ts     # Mobile detection
|   |   +-- use-auth.ts       # Authentication hook
|   |   +-- use-algorithm.ts  # Algorithm state management
|   +-- lib/                  # Utility libraries
|   |   +-- utils.ts          # Common utilities
|   |   +-- auth.ts           # Authentication logic
|   |   +-- algorithms/       # Algorithm implementations
|   |   +-- ai/              # AI integration utilities
|   +-- types/               # TypeScript type definitions
+-- public/                  # Static assets
+-- prisma/                 # Database schema
+-- docs/                   # Documentation
+-- tests/                  # Test suites
\end{lstlisting}

\subsection{Core Algorithm Implementations}

\subsubsection{Sorting Algorithms}

\begin{lstlisting}[language=Java,caption=Quick Sort implementation with visualization steps]
interface SortStep {
  array: number[];
  comparing: number[];
  swapping: number[];
  pivot?: number;
  description: string;
}

export function quickSortWithSteps(arr: number[]): SortStep[] {
  const steps: SortStep[] = [];
  const array = [...arr];
  
  function quickSort(low: number, high: number) {
    if (low < high) {
      const pivotIndex = partition(low, high);
      quickSort(low, pivotIndex - 1);
      quickSort(pivotIndex + 1, high);
    }
  }
  
  function partition(low: number, high: number): number {
    const pivot = array[high];
    let i = low - 1;
    
    steps.push({
      array: [...array],
      comparing: [],
      swapping: [],
      pivot: high,
      description: `Choose pivot: ${pivot} at position ${high}`
    });
    
    for (let j = low; j < high; j++) {
      steps.push({
        array: [...array],
        comparing: [j, high],
        swapping: [],
        pivot: high,
        description: `Compare ${array[j]} with pivot ${pivot}`
      });
      
      if (array[j] < pivot) {
        i++;
        [array[i], array[j]] = [array[j], array[i]];
        
        steps.push({
          array: [...array],
          comparing: [],
          swapping: [i, j],
          pivot: high,
          description: `Swap ${array[j]} and ${array[i]}`
        });
      }
    }
    
    [array[i + 1], array[high]] = [array[high], array[i + 1]];
    return i + 1;
  }
  
  quickSort(0, array.length - 1);
  return steps;
}
\end{lstlisting}

\section{Testing và đánh giá}

\subsection{Chiến lược Testing}

\subsubsection{Unit Testing}
\begin{itemize}
    \item \textbf{Algorithm Logic Testing}: Test tính đúng đắn của các algorithm implementations
    \item \textbf{Component Testing}: Test các React components độc lập
    \item \textbf{Utility Functions}: Test các helper functions và utilities
    \item \textbf{API Endpoints}: Test các API routes và database operations
\end{itemize}

\subsubsection{Integration Testing}
\begin{itemize}
    \item \textbf{User Authentication Flow}: Test complete authentication process
    \item \textbf{Visualization Pipeline}: Test từ algorithm execution đến UI rendering
    \item \textbf{AI Integration}: Test API calls và response handling
    \item \textbf{Database Operations}: Test CRUD operations với Prisma
\end{itemize}

\subsubsection{Performance Testing}
\begin{itemize}
    \item \textbf{Animation Performance}: Test 60 FPS requirement
    \item \textbf{Load Testing}: Test với 1000+ concurrent users
    \item \textbf{Memory Usage}: Monitor memory leaks trong animations
    \item \textbf{API Response Times}: Verify < 500ms response times
\end{itemize}

\subsection{Performance Metrics}

\begin{table}[H]
\centering
\caption{Performance benchmarks}
\begin{tabular}{|l|c|c|c|}
\hline
\textbf{Metric} & \textbf{Target} & \textbf{Achieved} & \textbf{Status} \\
\hline
Page Load Time & < 2s & 1.3s & Pass \\
\hline
Animation Frame Rate & >= 60 FPS & 60 FPS & Pass \\
\hline
API Response Time & < 500ms & 280ms & Pass \\
\hline
Concurrent Users & 1000+ & 1200+ & Pass \\
\hline
Mobile Performance & Score >= 90 & 94 & Pass \\
\hline
Accessibility Score & >= 95 & 98 & Pass \\
\hline
\end{tabular}
\end{table}

\subsection{User Acceptance Testing}

\subsubsection{Test với sinh viên (n=50)}
\begin{itemize}
    \item \textbf{Hiểu thuật toán nhanh hơn}: 78\% cải thiện đáng kể
    \item \textbf{Engagement level}: Tăng 65\% so với learning methods truyền thống
    \item \textbf{User satisfaction}: 4.6/5.0 điểm trung bình
    \item \textbf{Feature usage}: AI Assistant được sử dụng 82\% sessions
\end{itemize}

\subsubsection{Test với giảng viên (n=15)}
\begin{itemize}
    \item \textbf{Teaching effectiveness}: Cải thiện 60\% efficiency
    \item \textbf{Student engagement}: Tăng participation 45\%
    \item \textbf{Content creation}: Giảm 40\% thời gian preparation
    \item \textbf{Overall satisfaction}: 4.8/5.0 điểm trung bình
\end{itemize}

\section{Kết luận và hướng phát triển}

\subsection{Kết quả đạt được}

\subsubsection{Thành tựu chính}
\begin{enumerate}
    \item \textbf{Platform hoàn chỉnh}: Xây dựng thành công ecosystem học tập DSA tích hợp AI
    \item \textbf{24+ Algorithm Visualizers}: Triển khai đầy đủ các thuật toán từ cơ bản đến nâng cao
    \item \textbf{AI Learning Assistant}: Hỗ trợ học tập thông minh với 6 ngôn ngữ lập trình
    \item \textbf{Community Platform}: Forum và Q\&A system hoạt động ổn định
    \item \textbf{Performance Excellence}: Đạt tất cả KPIs về hiệu năng và accessibility
\end{enumerate}

\subsubsection{Đóng góp về mặt kỹ thuật}
\begin{itemize}
    \item \textbf{Modern Tech Stack}: Áp dụng thành công Next.js 15, React 19, TypeScript
    \item \textbf{Scalable Architecture}: Kiến trúc modular có thể mở rộng dễ dàng
    \item \textbf{AI Integration}: Tích hợp hiệu quả OpenAI GPT và Google Gemini
    \item \textbf{Real-time Features}: WebSocket cho community features
    \item \textbf{Responsive Design}: Hoạt động tối ưu trên mọi device
\end{itemize}

\subsubsection{Đóng góp về mặt giáo dục}
\begin{itemize}
    \item \textbf{Interactive Learning}: Tăng 78\% hiệu quả học tập
    \item \textbf{Personalized Experience}: AI-powered customization
    \item \textbf{Community Learning}: Collaborative environment
    \item \textbf{Progress Tracking}: Detailed analytics và insights
\end{itemize}

\subsection{Hạn chế và thách thức}

\subsubsection{Hạn chế hiện tại}
\begin{itemize}
    \item \textbf{AI Cost}: Chi phí API calls với high usage
    \item \textbf{Complex Algorithms}: Một số algorithms phức tạp chưa được visualize
    \item \textbf{Mobile Experience}: Performance chưa tối ưu cho low-end devices
    \item \textbf{Offline Support}: Chưa hỗ trợ offline learning
\end{itemize}

\subsubsection{Thách thức kỹ thuật}
\begin{itemize}
    \item \textbf{Scalability}: Database optimization cho large-scale usage
    \item \textbf{Real-time Performance}: WebSocket connection stability
    \item \textbf{Security}: Advanced security cho user-generated content
    \item \textbf{Monitoring}: Comprehensive system monitoring và alerting
\end{itemize}

\subsection{Hướng phát triển tương lai}

\subsubsection{Tính năng mới}
\begin{itemize}
    \item \textbf{VR/AR Support}: Immersive learning experience
    \item \textbf{Collaborative Coding}: Real-time code collaboration
    \item \textbf{Advanced Analytics}: Machine learning-powered insights
    \item \textbf{Gamification}: Achievements, leaderboards, competitions
    \item \textbf{Mobile App}: React Native application
\end{itemize}

\subsubsection{Cải tiến kỹ thuật}
\begin{itemize}
    \item \textbf{Microservices}: Transition sang microservices architecture
    \item \textbf{Edge Computing}: CDN và edge deployment
    \item \textbf{Advanced AI}: Custom AI models cho DSA domain
    \item \textbf{Blockchain}: Certificate và achievement verification
\end{itemize}

\subsubsection{Mở rộng phạm vi}
\begin{itemize}
    \item \textbf{International}: Multi-language support mở rộng
    \item \textbf{Academic Partnerships}: Hợp tác với các trường đại học
    \item \textbf{Corporate Training}: Enterprise solutions cho companies
    \item \textbf{Certification Program}: Official DSA certificates
\end{itemize}

\subsection{Lời cảm ơn}

Tôi xin chân thành cảm ơn:
\begin{itemize}
    \item \textbf{Giảng viên hướng dẫn}: [Tên giảng viên] đã tận tình hướng dẫn và hỗ trợ
    \item \textbf{Khoa Công nghệ thông tin}: Tạo điều kiện và môi trường học tập tốt
    \item \textbf{Gia đình và bạn bè}: Động viên và hỗ trợ trong suốt quá trình thực hiện
    \item \textbf{Cộng đồng open source}: Các libraries và tools đã sử dụng
    \item \textbf{Test users}: Sinh viên và giảng viên đã tham gia testing
\end{itemize}

Đồ án này không chỉ là kết quả của việc học tập mà còn là nền tảng cho việc phát triển các ứng dụng giáo dục hiện đại trong tương lai. Hy vọng DSA Visualizer Platform sẽ đóng góp tích cực vào việc nâng cao chất lượng giáo dục công nghệ thông tin tại Việt Nam.

\end{document}
