\documentclass[tikz,border=10pt]{standalone}
\usepackage[utf8]{inputenc}
\usepackage[T1]{fontenc}
\usepackage[vietnamese]{babel}
\usepackage{tikz}
\usetikzlibrary{shapes.geometric, arrows, positioning, shapes.multipart, fit, calc}

\tikzset{
    actor/.style={
        draw,
        rectangle,
        minimum width=1.5cm,
        minimum height=0.8cm,
        text centered,
        font=\scriptsize
    },
    lifeline/.style={
        draw,
        dashed,
        gray
    },
    message/.style={
        draw,
        ->,
        font=\tiny
    },
    selfmessage/.style={
        draw,
        ->,
        font=\tiny
    },
    activation/.style={
        draw,
        rectangle,
        fill=white,
        minimum width=0.3cm,
        minimum height=1cm
    }
}

\begin{document}
\begin{tikzpicture}[node distance=4cm]

% Title
\node[above] at (0,12) {\Large \textbf{Sequence Diagram - Algorithm Visualization Process}};

% ===== ACTORS AND OBJECTS =====
\node[actor] (student) at (-8,10) {Student\\User};
\node[actor] (ui) at (-3,10) {React\\UI};
\node[actor] (visualizer) at (2,10) {Algorithm\\Visualizer};
\node[actor] (ai) at (7,10) {AI\\Assistant};
\node[actor] (database) at (12,10) {Database\\Layer};

% ===== LIFELINES =====
\draw[lifeline] (-8,9.3) -- (-8,-1);
\draw[lifeline] (-3,9.3) -- (-3,-1);
\draw[lifeline] (2,9.3) -- (2,-1);
\draw[lifeline] (7,9.3) -- (7,-1);
\draw[lifeline] (12,9.3) -- (12,-1);

% ===== ACTIVATION BOXES =====
\node[activation] (act1) at (-3,8.5) {};
\node[activation] (act2) at (2,7.8) {};
\node[activation] (act3) at (7,7.1) {};
\node[activation] (act4) at (2,6.4) {};
\node[activation] (act5) at (-3,5.7) {};
\node[activation] (act6) at (2,5) {};
\node[activation] (act7) at (7,4.3) {};
\node[activation] (act8) at (12,3.6) {};
\node[activation] (act9) at (2,2.9) {};
\node[activation] (act10) at (-3,2.2) {};

% ===== MESSAGE SEQUENCE =====
% 1. Student selects algorithm
\draw[message] (-8,9) -- (-3,9) node[midway,above,font=\tiny] {1: selectAlgorithm("QuickSort")};

% 2. UI initializes visualizer
\draw[message] (-3,8.5) -- (2,8.5) node[midway,above,font=\tiny] {2: initializeVisualizer(algorithmType)};

% 3. Visualizer requests AI explanation
\draw[message] (2,8) -- (7,8) node[midway,above,font=\tiny] {3: requestExplanation(algorithm)};

% 4. AI generates explanation
\draw[message] (7,7.5) -- (2,7.5) node[midway,below,font=\tiny] {4: return explanationData};

% 5. Visualizer loads algorithm
\draw[message] (2,7) -- (2,6.5) node[right,font=\tiny] {5: loadAlgorithmSteps()};

% 6. Visualizer updates UI
\draw[message] (2,6.5) -- (-3,6.5) node[midway,above,font=\tiny] {6: updateVisualization(data)};

% 7. Student starts visualization
\draw[message] (-8,6) -- (-3,6) node[midway,above,font=\tiny] {7: startVisualization()};

% 8. UI starts animation
\draw[message] (-3,5.5) -- (2,5.5) node[midway,above,font=\tiny] {8: beginAnimation()};

% 9. Visualizer asks AI for step explanation
\draw[message] (2,5) -- (7,5) node[midway,above,font=\tiny] {9: explainCurrentStep(stepIndex)};

% 10. AI provides step explanation
\draw[message] (7,4.5) -- (2,4.5) node[midway,below,font=\tiny] {10: return stepExplanation};

% 11. Visualizer saves progress
\draw[message] (2,4) -- (12,4) node[midway,above,font=\tiny] {11: saveProgress(userId, stepData)};

% 12. Database confirms save
\draw[message] (12,3.5) -- (2,3.5) node[midway,below,font=\tiny] {12: return saveConfirmation};

% 13. Visualizer completes step
\draw[message] (2,3) -- (2,2.5) node[right,font=\tiny] {13: completeAnimationStep()};

% 14. UI updates display
\draw[message] (2,2.5) -- (-3,2.5) node[midway,above,font=\tiny] {14: updateStepDisplay()};

% 15. Display result to student
\draw[message] (-3,2) -- (-8,2) node[midway,above,font=\tiny] {15: showStepResult()};

% ===== NOTES =====
\node[draw,rectangle,fill=yellow!20,text width=3cm,font=\tiny] at (-8,-0.5) {
    \textbf{Note:}\\
    This sequence repeats\\
    for each algorithm step\\
    until completion
};

\node[draw,rectangle,fill=blue!20,text width=3cm,font=\tiny] at (7,-0.5) {
    \textbf{AI Integration:}\\
    Provides real-time\\
    explanations and\\
    adaptive learning
};

% 4. AI returns explanation
\draw[message] (6,5.5) -- (2,5.5) node[midway,above] {4: explanation};

% 5. Visualizer returns algorithm data
\draw[message] (2,5) -- (-2,5) node[midway,above] {5: algorithmData};

% 6. UI displays algorithm info
\draw[message] (-2,4.5) -- (-6,4.5) node[midway,above] {6: Hiển thị thông tin thuật toán};

% 7. Student inputs data
\draw[message] (-6,4) -- (-2,4) node[midway,above] {7: Nhập dữ liệu đầu vào};

% 8. Student starts visualization
\draw[message] (-6,3.5) -- (-2,3.5) node[midway,above] {8: Bắt đầu visualization};

% 9. UI starts visualization
\draw[message] (-2,3) -- (2,3) node[midway,above] {9: startVisualization(data)};

% 10. Visualizer generates steps
\draw[selfmessage] (2,2.5) -- ++(1,0) |- (2,2.3) node[midway,right] {10: Tạo animation steps};

% 11. Visualizer gets AI hints
\draw[message] (2,2) -- (6,2) node[midway,above] {11: getHints(currentStep)};

% 12. AI provides hints
\draw[message] (6,1.5) -- (2,1.5) node[midway,above] {12: hints};

% 13. Visualizer saves progress
\draw[message] (2,1) -- (10,1) node[midway,above] {13: saveProgress(userId, progress)};

% 14. Database confirms save
\draw[message] (10,0.5) -- (2,0.5) node[midway,above] {14: saved};

% 15. Visualizer returns animation
\draw[message] (2,0) -- (-2,0) node[midway,above] {15: animationData};

% 16. UI renders visualization
\draw[message] (-2,-0.5) -- (-6,-0.5) node[midway,above] {16: Hiển thị animation};

% 17. Student interacts with controls
\draw[message] (-6,-1) -- (-2,-1) node[midway,above] {17: Điều khiển animation};

% 18. UI updates visualization
\draw[message] (-2,-1.5) -- (2,-1.5) node[midway,above] {18: updateVisualization(action)};

% Notes
\node[draw, rectangle, text width=3cm, font=\tiny] at (-8,2) {
    \textbf{Ghi chú:}\\
    - Bước 1-6: Tải thuật toán\\
    - Bước 7-18: Chạy visualization\\
    - AI hỗ trợ giải thích và gợi ý
};

\end{tikzpicture}
\end{document}

